\chapter{[Chapter Title]}\label{ch:common}

This is the beginning of Chapter~\ref{ch:common}. Always have text between every head
and subhead.
We have reproduced all examples from NPS library~\cite{orend_2013,
stein_1994,krishnan_2008,abramowitz_64,orend_2013b,myer_77,
NPS_notes_2013,panda1,panda2,kirby_2013,janow_1994,delac_20014,
nsa_ipac_2012,dod_5000.1,instruction_1000.01,transmission_comm_85,
engineer_ops_2011,talleen_1996,wang_2014,semilof_1996,congress_1991,
ellwood_2013,martin_2014,wilkinson_1990,brooks_1992,alexander_2012,
reber_1988,ieee_308,kawasaki_1993,williams_1993,garcia_unpub,
kaneshiro_2011,vergun_2014,raytheon_2014,ge_2013,dept_vet_2013,
nps_tpo_2014,wiki_2016,youtube_2013,youtube_2014}.
Generally, the most commonly used references are
   books~\cite{IEEEexample:book_typical},
   journal articles~\cite{IEEEexample:article_typical},
   conference proceedings~\cite{IEEEexample:conf_typical},
   online resources~\cite{IEEEhowto:IEEEtranpage},
   theses~\cite{IEEEexample:masters},
   private communication~\cite{IEEEexample:private}, and
   class notes~\cite{NPS_notes_2013}.
Here are some citations\footnote{These citations were taken
from the IEEEtran example bib file.} that demonstrate nearly every bib 
style\footnote{\lipsum[10]} that exists~\cite{IEEEhowto:IEEEtranpage,
IEEEexample:shellCTANpage,IEEEexample:bibtexuser,IEEEexample:bibtexdesign,
IEEEexample:tamethebeast,IEEEexample:bibtexguide,
IEEEexample:article_typical,IEEEexample:articleetal,IEEEexample:conf_typical,
IEEEexample:book_typical,IEEEexample:articlelargepages,
IEEEexample:articledualmonths,IEEEexample:TBParticle,
IEEEexample:bookwitheditor,IEEEexample:book,IEEEexample:bookwithseriesvolume,
IEEEexample:inbook,IEEEexample:inbookpagesnote,IEEEexample:incollection,
IEEEexample:incollectionwithseries,IEEEexample:incollection_chpp,
IEEEexample:incollectionmanyauthors,
IEEEexample:motmanualhowpub,IEEEexample:confwithadddays,
IEEEexample:confwithvolume,IEEEexample:confwithpaper,
IEEEexample:confwithpapertype,IEEEexample:presentedatconf,
IEEEexample:masters,IEEEexample:masterstype,IEEEexample:phdurl,
IEEEexample:techrep,IEEEexample:techreptype,IEEEexample:techreptypeii,
IEEEexample:techrepstdsub,IEEEexample:unpublished,IEEEexample:electronhowinfo,
IEEEexample:electronhowinfo2,IEEEexample:electronorgadd,IEEEexample:uspat,
IEEEexample:jppat,IEEEexample:frenchpatreq,
IEEEexample:standard,IEEEexample:standardproposed,IEEEexample:draftasmisc,
IEEEexample:miscforum,IEEEexample:whitepaper,IEEEexample:datasheet,
IEEEexample:private,IEEEexample:miscrfc,IEEEexample:softmanual,
IEEEexample:softonline,IEEEexample:miscgermanreg,IEEEexample:bluebookstandard
}.

This shows the acronym macro being used for \ac{TCP}, which is
produced in its short form \ac{TCP} on all subsequent uses of the macro.
Acronyms are re-set after the abstract and executive summary, and will be
shown in their long form in their first use by default. You can mark some
special acronyms so they appear in their short form by default.

\section{A Section}\label{sec:something}
A simple equation to check numbering.
\begin{equation}
a^2 + b^2 = c^2
\end{equation}
\lipsum[1-4] % remove me

\begin{figure}
\framebox[\textwidth]{\parbox{\textwidth}{\lipsum[1]}} % example figure (text in a box)
\caption[Short figure title (customized for LoF).]{Short figure title.}
\caption*{\small This is the long caption that explains the figure in detail and
expounds on its relevance to the text.
All figures need to be referenced in the text before the image or table.
Full source citation, as applicable, is required.
Source~\cite[Figure 10]{IEEEexample:article_typical}: \bibentry{IEEEexample:article_typical}.}
\end{figure}

\section{Another Section}
A simple equation to check numbering.
\begin{equation}
L' = {L}{\sqrt{1-\frac{v^2}{c^2}}}
\end{equation}
\lipsum[2-3] % remove me

\subsection{A subsection}
\lipsum[5-6] % remove me

\section{Yet Another Section}
\lipsum[1] % remove me
\begin{table}
\caption{This is the less than 15 word title.}
\begin{center}
\begin{tabular}{ c c c }
\hline
  1 & 2 & 3 \\ \hline
  4 & 5 & 6 \\
  7 & 8 & 9 \\
  10 & 11 & 12 \\
  13 & 14 & 15 \\
\hline
\end{tabular}
\end{center}
\caption*{\small This is the long caption that explains the table.
This table is not entirely original and requires an ``Adapted from'' caption.
Adapted from~\cite[Table 5]{IEEEexample:article_typical}: \bibentry{IEEEexample:article_typical}.}
\end{table}
\lipsum[2]

\subsection{A subsection with a long name that continues onto two lines and should be single-spaced within the title}\label{sec:another}
\lipsum[3]

\subsubsection{A subsubsection}\label{sec:minorstuff}
\lipsum[4]

\section{Things to Remember When Writing}\label{sec:remember}
\begin{enumerate}
\item Punctuation (periods and commas) go inside quotation marks. 
\item The macros \verb+\etal+, \verb+\ie+, \verb+\eg+ and \verb+\etc+ force proper
  American English convention for these (\ie a comma follows).
  It is redundant and incorrect to use \etc at the end of a list of
  examples (\eg apples, pears, \etc).
\item Chicago style recommends subordinate clauses using \verb+\ie+ and \verb+\eg+ be
separated from the main clause using parentheses (\eg, as shown above).
\item Use the \LaTeX{} \verb+\begin{figure}+ and \verb+\begin{table}+ environment to
  create floating figures and tables. Use the \verb+\caption+ command
  to create your captions. Label your captions with the
  \verb+\label{foo}+ command inside the caption itself. Reference
  these figures and tables with the \verb+\ref{foo}+ reference command.
\item Do not split text around a figure or table. 
\item Master's degree has an apostrophe and Postgraduate is one word. 
\item Most acronyms do not need periods, like \ac{NPS}. There are some exceptions:
\ac{NPS} ``house-style'' prefers United States when used as a noun and the acronym (with periods) when used as an adjective (\eg, \ac{US} soil, \ac{US} forces).
Common acronyms do not appear in the list of acronyms (\eg, \ac{US}, \ac{FBI}, \ac{CIA}).
\item If you use ``however,'' make sure there's a comma before and after,
unless you start a sentence with it. However, it is best not to start a sentence
with ``however.'' And while we are on the subject, you should try to avoid starting a sentence with ``and'' or ``because.'' 
\item When typing a date, do not use ``st'' or ``th.'' Instead, just
  note the date: July 4, 1776, is Independence Day. Commas go 
after month/date, year: both Jefferson and Adams died on July 4, 1826.
No comma between month/yr: \textit{Alice's Adventures in Wonderland} was published in July 1865.
\item In general, spell out numbers one through nine, and use numerals for 10 and greater.
\item Use automatic numbering and lettering.
\item Capitalize C in Chapter, F in Figure, T in Table and E in Equation when referring
to chapters, figures, tables and equations in the text.
\item When there is more than one reference, put them both into the \verb+\cite+ command: \verb+\cite{john1,john2}+. It will render like this \cite{IEEEhowto:IEEEtranpage,IEEEexample:shellCTANpage}.
\item Avoid writing in the first person!
\item Make sure there are no widows at the
  top of the page---if \LaTeX{} gives you a hard time, you may need to
  add or remove text so that everything works out properly.
\item Footnote numbers go outside the punctuation. 
\item When typing equations in text, use ``where'' or ``if.'' Use
  Math Mode. 
\item When inserting symbols, use Math Mode.
\end{enumerate}
