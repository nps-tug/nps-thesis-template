%%
%% techreport.tex:
%% 
%% The source code for the NPS report that describes the LaTeX style
%%
\documentclass[twoside,singlespace,twoauthors]{npsreport}

%
% Put extra packages you may need to customize your thesis
%
\usepackage{listings}

%% Hyperref adds PDF metadata
%% TAKE THIS OUT if it causes problems.
%% Specifically, remove if you are using xelatex
\usepackage[pdftex,
  pdftitle={\@title}, 
  pdfauthor={\@author},
  pdfkeywords={\@SubjectTerms},
  pdfpagemode=UseOutlines,
  bookmarks,
  bookmarksopen=false,
  urlbordercolor=white,
  linkbordercolor=white,
  citebordercolor=white,
  filebordercolor=white,
  menubordercolor=white,
  runbordercolor=white]{hyperref}	% understand PDF refs
 \usepackage[all]{hypcap}               % must follow hyperref


\title{The New NPS \LaTeX\ Report and Thesis Format}

\author{Simson L.\ Garfinkel}
\authortwo{Travis W. Axtell}

\distribution{Approved for public release; distribution is unlimited}

\abstract{This report introduces the NPS \LaTeX{} templates
  that can be used to produce a master's thesis, a
  Ph.D.\ dissertation, or a technical report. The template can produce
  documents that are unclassified, For Official Use Only (FOUO), or classified. Additional information on using \LaTeX\,is also provided.}

%
% Mandatory fields for the SF298.
%
\ReportType{Technical Report}
\DatesCovered{2008-01-01---2011-09-31}
\SponsoringAgency{Department of the Navy}
\ReportClassification{Unclassified}
\AbstractClassification{Unclassified}
\PageClassification{Unclassified}
%
% Optional fields for the SF298.
%
\RPTpreparedFor{}
\SponsoringNotation{}
\ContractNumber{}
\GrantNumber{}
\ProgramElementNumber{}
\TaskNumber{}
\WorkUnitNumber{}
\POReportNumber{NPS-CS-11-011}
\Acronyms{}
\SMReportNumber{}
\SubjectTerms{NPS Thesis, NPS Technical Report}
\ResponsiblePerson{}
\RPTelephone{}
\SignatureOne{}
\SignatureTwo{}
\SupplementaryNotes{The views expressed in this document are those of
  the author and do not reflect the official policy or position of the
  Department of Defense or the U.S. Government. %
  IRB Protocol Number: N/A % if you need to note an IRB Protocol or N/A
}

% Optional. Prevents footnotes from being reset at each chapter
% Comment this out to have them reset with each chapter.
\makeatletter
\@removefromreset{footnote}{chapter}
\makeatother

%
% Your thesis begins here
%
\begin{document}

\NPScover                       % pretty NPS cover page
\NPSsignature                   % NPS signature page
\NPSsftne                       % NPS SF298
%\NPSthesistitle            % Title page
%\NPSabstractpage           % Abstract Page
\NPSfrontmatter            % NPS front matter follows

\NPStableOfContents
%\NPSlistOfFigures
%\NPSlistOfTables
%\NPSlistOfAcronymsFromFile{acronyms}

%
% Put Executive summary here.
% New paragraphs start after an empty line.
%
%\NPSexecsummary{
%  \lipsum[1-4] % example text; remove me
%}

%
% Put acknowledgements here.  
% New paragraphs start after an empty line.
%
%\NPSacknowledgements{
%  \lipsum[1-3] % example text; remove me
%}

% Start layout for the NPS body
\NPSbody                        % Set up for NPS Body

%% Some useful stuff that we do locally
\DefineShortVerb{\|}            % makes |foo| a verbatim command

% a command for making comments, we use it in the listings environment
\newcommand{\comment}[1]{\hfill \\ \; \normalfont\itshape\footnotesize\color{blue!70} $\Rightarrow$ #1}
\newcommand{\caution}[1]{\textcolor{red}{\emph{#1}}}


% settings for the listing environment
\lstset {
  language=TeX,
  frame=bt,
  xrightmargin=1mm,
  escapeinside={(*@}{@*)},
  numbers=left,
  basicstyle={\scriptsize\ttfamily},
  aboveskip=3mm,
  belowskip=3mm,
  showstringspaces=false,
  commentstyle={\color{black!80}},
  breaklines=true,
  breakatwhitespace=true,
  tabsize=4,
  moredelim=*[s][\ttfamily]{:}{:} %Newly added line
}

% CHAPTERS
% You have two options on how to structure your thesis:
% a) A single file. All chapters, sections, etc. go in this file.
%    This can make navigating your thesis a little more difficult.
% b) Use multiple files.  One chapter per file is recommended.
%    This breaks your thesis up into logical units to edit.
%
\chapter{Introduction to \LaTeX}
\LaTeX{} is a text formatting system created by Leslie Lamport in the early 1980s\cite{latex}. The program is based on the \TeX{} text formatting system created by Donald Knuth in 1978\cite{tex}. With 
\LaTeX{} you author your document by editing a text input file using a
program such as EMACS, |vim|, or another editor. You then give
this input file to \LaTeX{} (or, more accurately, to a program such as
|pdflatex| or |xelatex|). \LaTeX{} then transforms your
input file(s) into an Adobe Portable Document Format (PDF) file.

Although most documents at NPS are prepared with
Microsoft\textregistered{} Word, \LaTeX{} is widely used outside of
NPS in the sciences to create conference papers, journal articles, and
even full-length books. \LaTeX{} is especially popular in computer
science. With the NPS template you can use \LaTeX{} to produce an NPS
thesis that is consistent with NPS formatting requirements. Because
\LaTeX{} provides for automated formatting, automatic updating of
references, and the ability to directly embed experimental results,
many students who write technical documents at NPS that use \LaTeX{}
find that they save time---even when they take into account the time
that they spend learning to use \LaTeX{} in the first place!

\section{Reasons to use \LaTeX}
There are many reasons to use \LaTeX{} for preparing
a technical document:

\begin{itemize}
\item Formatting for paragraphs, quotations, lists, tables, and other
  elements is performed automatically. You can easily make changes to
  formatting and have them reflected throughout your document.  The
  numbers used for numbered sections are automatically updated when
  new material is added or removed. As a result, formatting is more
  consistent with \LaTeX{} than with Microsoft Word.

\item References within your document to numbered chapters,
  appendices, sections, figures, tables, equations and references are
  updated automatically each time a |.pdf| file is produced, assuring
  that they are  correct.

\item The Bib\TeX{} system produces consistent citations and
  bibliography. References are pulled from a bibliographic database
  that is separately maintained and can be shared between many documents. Records for the bibliographic
  database can be downloaded from many online services, helping to
  assure that they are consistent. The citation format is maintained 
  separately from the citation contents, making it easy to change
  citation styles when submitting to different conferences or journals.  
\item \LaTeX{} allows you to directly include other files at the time
  that the |.pdf| file is created. This makes it easy to automatically
  incorporate the results of experiments in tabular or graphical form,
  without having to manually copy results.  Source code for programs
  can be included with pretty syntax from the original files, and line
  numbers can be automatically displayed as desired. 
\item |pdflatex|, the version of \LaTeX{} we recommend at NPS, also
  allows you to embed other files as attachments within your |.pdf| file. This makes it
  easy to preserve  experimental data, spreadsheets, or other
  information in the final file that is distributed to sponsors and archived. 
\item Because the \LaTeX{} input file is plain ASCII, you can store
  your document using a revision control system such as
  Subversion (SVN)\cite{subversion}. This allows multiple people to work on
  the same document at the same time; Subversion automatically merges
  the changes together.  If you save your thesis work daily, it is possible
  retrieve previous revisions of your thesis and undo changes or
  mistakes---even many months after the fact.
\item \LaTeX{} is free software and runs on PCs, Macs, and Unix
  systems. This means that you can produce your documents on
  practically every computer you have, without having to purchase
  anything else.
\end{itemize}

NPS has developed this template for preparing NPS reports and theses. If
you are using \LaTeX{} at NPS, there are many reasons for using the
template:

\begin{itemize}
\item The template has been in use since 2007. Numerous NPS
  faculty members have worked with the NPS Thesis Processor to assure
  that the documents produced are acceptable for student and faculty
  use.
\item The template supports the creation of unclassified, For Official Use Only (FOUO), and
  classified documents. In particular, the template supports
  classification labels for paragraphs, captions, and references.
\item The cover page, signature page, and Standard Form 298 are automatically generated.
\item Combined with the Subversion, the template makes it possible for students
  and their advisors to collaborate on a document without the need to pass files back and forth. 
\end{itemize}


\section{The Purpose of this Document}

This document does not duplicate the depth of information
available elsehwere on \LaTeX{},
but does provide you with the minimum amount of information required to use \LaTeX{}
to produce a master's thesis or technical report at NPS.

If you wish to learn more about \LaTeX, there are many good reference
books and online tutorials for \LaTeX{}.

We recommend these online resources:
\begin{enumerate}
\item \url{http://en.wikibooks.org/wiki/LaTeX}, a wikibook that
  discusses many aspects of \LaTeX.
\item \url{http://www.ctan.org/}, the Comprehensive \TeX{} Archive
  Network, a collection of modules and documentation for extending \LaTeX{}.
\item \url{http://en.wikipedia.org/wiki/Comparison_of_TeX_editors}, a
  web page on Wikipedia that discusses different programs available for editing \LaTeX{} input files.
\item \url{http://tex.stackexchange.com/}, a question-and-answer website for \TeX{} and \LaTeX{} problems.
\end{enumerate}

We also recommend these books. They are expensive, but worth if it if
you wish to become a \LaTeX{} master.
\begin{enumerate}
\item Guide to \LaTeX{} (4th Edition), by Patrick W. Daly. This is the
  comprehensive \LaTeX{} reference which will provide you with an
  astounding amount of \LaTeX-related information. Do not read
  Lamport's original book, as it is quite out of date.
\item The \LaTeX{} Companion (Tools and Techniques for Computer
  Typsetting), by Frank Mittelbach, Michel Gossens, Johannes Braams,
  David Carlisle and Chris Rowley.
\item The \LaTeX{} Graphics Companion, by Michael Gossens, Frank
  Mittelbach, Sebastian Rahtz, and Denis Roegel.
\end{enumerate}

\section{Installation}
Before you can use \LaTeX{}, you will need to install two critical
pieces of software:
\begin{enumerate}
\item The \LaTeX{} system itself.
\item A program for editing the \verb+.tex+ input files.
\end{enumerate}

Here once again there are many options. For both \LaTeX{}
and text editors there are both free  and commercial
distributions. This document makes specific recommendations that were
known to work as of the document's date of publication. You are free
to explore on your own as well.

\subsection{Installation on MacOS 10.5 and above}

The easiest way to get \LaTeX{} operational on a Mac is
 to download an installer for the most recent distribution
 from the \TeX Users's Group (TUG) at \url{http://www.tug.org/mactex}. 

TUG's Mac\TeX{} distribution will install \LaTeX{} in the
\verb+/usr/texbin/pdflatex+ directory and will update your startup files to
include this directory in your path. If you chose this strategy, be
sure to click ``Customize'' in the installer and select  all of
the optional packages for installation.

You can also install the program from sources using the MacPorts or Fink installer system, but this is not recommended. 

If you are using \LaTeX{} on a Mac, you should consider downloading
and using \emph{LaTeXiT}, an open-source program that allows you to type
\LaTeX{} math and create PDFs for embedding in other applications. The
created PDFs have the source \LaTeX{} code embedded, so you can re-edit
them in LaTeXiT using the Mac ``Services'' feature.  You may also consider the
commercial program \emph{Latexian}, which allows you to type a
\LaTeX{} document in one window
and see the PDF update in another window as you type.

\subsection{Installing on Linux}
For most Linux systems a complete \LaTeX{} distribution 
can be downloaded as part of the |texlive-latex3| package. This
package can be downloaded in source from the \url{http://tug.org/}
website and compiled locally, or installed directly using a package
management command. In either case you will need to the installation
as the superuser; this is typically done with the |sudo| command.

For Debian and Ubuntu Linux, use the |apt-get| command:

\begin{Verbatim}
% sudo apt-get install texlive-latex3
\end{Verbatim}

For RedHat and Centos, use the |yum| command:

\begin{Verbatim}
% sudo yum install texlive-latex3
\end{Verbatim}

We have noticed that the install command occasionally fails. If it
fails for you, try it again two or three more times. If that still
does not work, you will need to download and install from source.

While the Fedora distribution includes an old version of
\TeX Live from 2007, we don't advise using it.  Fortunately, there is
an ongoing official project keeping an up-to-date release of \TeX
Live available for Fedora.  Directions for installing it are at
\url{http://fedoraproject.org/wiki/Features/TeXlive}.  Eventually,
this release will be merged back to the main Fedora distribution.

\subsection{Installation on Microsoft Windows}
\TeX Live is the most up-to-date distribution for Windows and can
be downloaded from \url{htp://www.tug.org/texlive/}.  The
distribution includes the {\TeX}Works editor for working with
\LaTeX{} documents.  You can also use the {\TeX}lipse plugins for
the Eclipse IDE at \url{http://texlipse.sourceforge.net/}.

Another of the popular distributions for Microsoft Windows machines
is called MiK\TeX, which can be downloaded from
\url{http://www.miktex.org/}.  In most cases, this distribution of
\LaTeX{} will automatically download additional packages if needed.
At this time, Mik\TeX is lagging \TeX Live in frequency of updates.

\subsection{Creating classified documents}
If you are producing a classified thesis, you should ask a
system administrator at the STBL or SCIF to provide you with a
\LaTeX{} installation on a computer authorized to handle classified
information. With appropriate approvals you can alternatively set up an
installation on a stand-alone machine.
In the event that you are missing a package to create your thesis, the |.sty| files can be copied from the package on the CTAN website into your thesis directory.  A system administrator should be consulted on the file transfer process.  

\section{Additional Applications for \LaTeX{} }

In addition to downloading and installing \LaTeX, you will need an
editor for editing the |.tex| input files. These programs are
sometimes called \emph{front ends}, although the term is imprecise and
probably incorrect. You can use \emph{any} editor for editing a |.tex|
file, even Microsoft Word. Indeed, most of the front ends for \LaTeX{} 
are really just text editors with syntax highlighting, although some
will automatically compile your document and even jump to errors in
the input file when they are encountered.

\subsection{Selection of an Editor} Text editors offer a variety of features.
Some are easier for beginners, such as Notepad++, LEd, and TeXnicCenter.
Advanced editors for skilled users include |emacs|, |vim|, and |Texlipse|.
Some recent TeX editors that have been gathering attention are |TeXstudio|,
|TeXMaker|, and |Sublime Text| (with LaTeXing package).  A thorough comparison
of editors is available at
\url{http://en.wikipedia.org/wiki/Comparison_of_TeX_editors}.  Learning about
the various features of the text editors can help you to dramatically improve
your writing efficiency.

If you are running \LaTeX{} on MacOS, you already have a powerful editor
installed on your computer: EMACS, which can be run from within the
Macintosh Terminal application. You can run the tutorial for EMACS by
starting Terminal, typing |emacs| and enter at the command prompt, and
then typing control-h followed by a ``|t|'' to start the tutorial.

\subsection{LyX: An alternative \LaTeX{} system}

LyX is a program that provides a WYSIWYG (What You See Is What
  You Get) graphical user interface for \LaTeX. Instead of editing the
|.tex| file directly, however, you edit an intermediate form which is
a restricted set of \LaTeX. LyX then runs \LaTeX{} for you and
produces the results.

While LyX is easier to use than \LaTeX, it does not have the power or
the flexibility.

LyX is free and open source and actively supported. LyX provides a GUI interface for floating figures and tables,
formatting, fonts, labels, chapters, sections, subsections, math
equations, tables, and much more.  It also has built-in features for
supporting Bib\TeX{} bibliographies, citations, and cross-references,
and generally supports anything that can be done with \LaTeX{}.

LyX can be downloaded from  \url{http://www.lyx.org/} and is available
for Linux, MacOS and Windows.  If you want spell-checking, you will
also need to install the \emph{aspell} package. 

An NPS thesis and dissertation template developed by CDR Michael
Bilzor is available for use with LyX. It can be be obtained from
\url{http://simson.net/npsthesis/lyxthesis.zip}.  

Further information about the LyX template can be found in the Appendix.


\section{Running \LaTeX}\label{runninglatex}
The \LaTeX{} system is actually a set of programs. For creating a thesis at NPS you will use several programs:

\begin{description}
\item{\texttt{pdflatex}} This program reads the input file (\eg
  |thesis.tex|) and produces a |.pdf| file (\eg |thesis.pdf|)
  as an output. This program also produces a number of intermediate
  files (|thesis.aux|, |thesis.bbl|, |thesis.toc|,
  \etc).
\item{\texttt{xelatex}} is a version of \LaTeX designed to process
  Unicode used in non-Roman languages. In some cases packages that are
  designed to work with \LaTeX will not work with \texttt{xelatex},
  which is why we do not recommend using it unless you have no other choice.
\item{\texttt{latex}} This is an older version of the |pdflatex|
  program that produces |.dvi| files. The |.dvi| file must then be
  transformed into either a |.ps| or a |.pdf| file. In practice you
  should not run |latex| unless you need to use a special graphics
  module called |PStricks|. That module is beyond the scope of this document.
\item{\texttt{bibtex}} This program reads the |thesis.bbl| file and
  produces a bibliography in a file called |thesis.bst| which
  includes the bibliography. The |thesis.bst| then gets read the
  next time |pdflatex| is run.
\item{\texttt{authorindex.pl}} This is a program in perl that produces the
  author index from the |thesis.bbl| file. The authorindex is
  saved in the file |thesis.ain|.
\item{\texttt{fixerrors.py}} It turns out that there is a bug in Bib\TeX{}
  which causes URLs longer than 53 characters to be improperly
  split. This program unsplits them. It also will correct |authorindex| errors.  You do not need to use this
  program if you do not have these errors.
\end{description}

If you are processing a file \emph{thesis.tex} to create a
\emph{thesis.pdf} file, you will typically execute these commands in
this order:

\begin{enumerate}
\item |pdflatex thesis|
\item |bibtex thesis|
\item |python fixerrors.py thesis|
\item |pdflatex thesis|
\item |pdflatex thesis|
\end{enumerate}

The first run of |pdflatex| creates the file |thesis.aux| (and any
other |.aux| files that might be needed).  A PDF file is also created,
but if you have any backwards references in your document the PDF file
will contain incorrect references. The subsequent runs read
these |.aux| files and generate correct back-references. Some editors, such as {\TeX}lipse, will run these commands as needed on behalf of the user.  Each command has a |thesis| argument, which is
 referring to the file |thesis.tex| without the |.tex| file extension.
 In rare cases it may be necessary to delete the |.aux| files and
 re-run the |pdflatex| command from the beginning.

\section{Basic \LaTeX{} formatting}
Here is a simple \LaTeX{} document:
\begin{Verbatim}
\documentclass{article}
\begin{document}
Hello World!
\end{document}
\end{Verbatim}

Normally with \LaTeX{}, you just type text and leave a blank line between
each paragraph. \LaTeX{} then formats it into beautiful
paragraphs. \LaTeX{} will ignore the space at the beginning of each line.

Here is a slightly more complex document:

\begin{Verbatim}
\documentclass{article}
\begin{document}
In December 1951, in a move virtually unparalleled in the history of
academe, the Postgraduate School moved lock, stock and wind tunnel
across the nation, establishing its current campus in Monterey,
Calif. 
     The coast-to-coast move involved 500 students, about 100
     faculty and staff and thousands of pounds of books and 
     research equipment. Rear Adm.\ Ernest Edward Herrmann 
     supervised the move that  pumped new vitality into the
     Navy's efforts to advance naval science and technology.

% This is comment. Nobody will see it.

Today the school, known as the ``Naval Postgraduate School,'' is the
Navy's preeminent institution of graduate education and advanced
research. Approximately 1 in 10 of the students are in the top 10\% of
their classes.
\end{document}
\end{Verbatim}


\LaTeX{} will format the above text into a document that looks  like this:

\fbox{
\begin{minipage}{6in}
\setlength{\parindent}{2pc}
In December 1951, in a move virtually unparalleled in the history of
academe, the Postgraduate School moved lock, stock and wind tunnel
across the nation, establishing its current campus in Monterey,
Calif. The coast-to-coast move involved 500 students, about 100
faculty and staff and thousands of pounds of books and research
equipment. Rear Adm.\ Ernest Edward Herrmann supervised the move that
pumped new vitality into the Navy's efforts to advance naval science
and technology.\\
\\
\indent Today the school, known as the ``Naval Postgraduate School,'' is the
Navy's preeminent institution of graduate education and advanced research.
Approximately 1 in 10 of the students are in the top 10\% of
their classes.
\end{minipage}
}

This sample document illustrates a few important points about \LaTeX:
\begin{itemize}
  \item \LaTeX{} will automatically
  re-wrap your text as necessary to format the paragraphs. Indention is determined by the style of the current document, not by whether or not you actually indent the paragraph.
\item \LaTeX{} ignores space at the beginning of lines; breaks between
  paragraphs are marked with blank lines.
\item What you type is not what you get!  In particular, opening
  double quote marks are typed as two backquotes (\verb+``+) and closing
  double quote marks are entered as two apostrophes (\verb+''+). You will
  also note that the period following Rear Adm.\ Ernest Edward
  Herrmann's name is followed by a backslash (\verb+\+) and a space, rather
  than just a space. This tells \LaTeX{} that the period does not mark
  the end of a sentence.
\item Commands begin with a backslash (|\|) and contain only uppercase
  and lowercase letters.
\item A command can have zero or more arguments. The arguments are
  enclosed within braces (|{}|). The |\documentclass| command begins
  the document; its argument is the kind of document you are making. 
\item The |\begin| command introduces an \emph{environment}. Every
  document has at last one environment, the |document|
  environment. Every |\begin| must have a matching |\end| that names
  the same environment. Environments can be nested.
\item Comments can be embedded in your document with a percent sign
  (|%|). Anything after the percent sign will not print. To print a
  percent sign, prefix it with a backslash (|\%|).
\end{itemize}

\section{Typing Special Characters}
This section provides information on how to type special characters in
\LaTeX. In each section we will have a table that shows what to type
and how it displays in your final |.pdf| file.  If the above text were
put into a table, it would look like this:

\begin{center}
\begin{tabular}{l|l||l|l}
Typed & Displayed & Typed & Displayed \\\hline
|``|  & ``        & |''|  & '' \\
\end{tabular}
\end{center}

The left entry of the table is the backquotes, which shares the tilde key on the US keyboard.
The right entry is the single quotes, which shares the double quote marks key.

\subsection{Typing Quotes}

To type quotes, you should not use the double-quote character. Instead,
use the back quote (|`|) and the forward quote (|'|)
to type quotes:

\begin{center}
\begin{tabular}{l|l||l|l}
Typed               & Displayed & Typed   & Displayed\\\hline
|don't|             & don't     &  |3'2''|  & 3'2''\\
|``this''|          & ``this''  & |`is'|  & `is' \\
|``\,`special'\,''| & ``\,`special'\,'' \\
\end{tabular}
\end{center}


\subsection{Controlled Special Characters}\label{special}

Unlike Microsoft Word and other programs, \LaTeX{} uses special
characters embedded in your text to control formatting. The most
common of these characters is the backslash ($\backslash$). All of
special characters are listed below:

\begin{center}
\begin{tabular}{c|p{4in}}
Special Character & Why it is special\\
\hline
|\|      & Introduces a command\\
\verb|{| & Introduces arguments in commands or the start of a group\\
\verb|}| & Closes arguments in commands or the end of a group\\
|%|      & The comment character; \LaTeX{} ignores the rest of the line\\
|#|      & Used for parameter substitution inside macros\\
|~|      & Enters a hard, non-breaking space\\
|&|      & Used for delimiting columns in a table \\
\verb|$| & Turns on/off math mode (see \S\ref{math})\\
|_|      & Used for subscript in math mode\\
|^|      & Used for superscript in math mode\\
\end{tabular}
\end{center}

To enter the special characters into your
document you must use a special sequence that begins with a
backslash. Most (but not all) of these special sequences are the
character itself. If you are curious, inside \LaTeX{}, each of these
sequences is implemented as a command that causes \LaTeX{} to output the
character that has been quoted:

\begin{center}
\begin{tabular}{l|l||l|l}
Typed  & Displayed & Typed          & Displayed\\\hline
|\$|   & \$        & |\&|           &    \&   \\
|\{|   & \{        & |\}|           &    \}   \\
|\%|   & \%        & |\_|           &    \_   \\
|\#|   & \#        & |\^{}|         &    \^{} \\
|\~{}| & \~{}      & |$\backslash$| & $\backslash$ \\
\end{tabular}
\end{center}


\subsection{Accented, Dotless and Slashed Vowels}
With \LaTeX{} most accented vowls are entered with a combining accent character and a
vowel, although some (such as the angstrom symbol) are not, as shown below:

\begin{center}
\begin{tabular}{l|l||l|l}
Typed    & Displayed & Typed   & Displayed\\\hline
|\'{o}|  & \'{o}     & |\~{o}| & \~{o} \\
|\'{o}|  & \'{o}     & |\={o}| & \={o} \\
|\^{o}|  & \^{o}     & |\.{o}| & \.{o} \\
|\"{o}|  & \"{o}     & |\d{o}| & \d{o} \\
|\c{o}|  & \c{o}     & |\u{o}| & \u{o} \\
|\b{o}|  & \b{o} \\
|\aa|    & \aa       & |\AA|   & \AA \\
|\i|     & \i        & |\j|    & \j \\
|\o|     & \o        & |\O|    & \O \\
|\ae|    & \ae       & |\AE|   & \AE \\
|\oe|    & \oe       & |\OE|   & \OE \\
|\v{o}|  & \v{o}     & |\H{o}| & \H{o} \\
|\t{oo}| & \t{oo} \\
\end{tabular}
\end{center}

Note that |\i| displays a dotless i while |\j| displays a dotless
j. Some fonts do not have some of these characters, and display a
black box instead.

\subsection{Symbols}
\LaTeX{} has literally hundreds of symbols that you can include in
your document. These symbols  include the copyright symbol, currency symbols, foreign language characters, and many more.  
The symbols are placed in documents using macros, allowing 
a plain text document to support a large variety of non-standard text characters.  A
complete guide to the available symbols in \LaTeX{} is available online at 
\url{http://mirror.ctan.org/info/symbols/comprehensive/symbols-letter.pdf} .

Here are some of the most common symbols you are likely to need for an
NPS document:

\begin{center}
\begin{tabular}{l|l||l|l}
Typed & Displayed & Typed & Displayed\\\hline
|\l|     & \l    & |\L|         & \L \\
|\S|     & \S    & |\ss|        & \ss \\
|\P|     & \P    & |\pounds|    & \pounds \\
|?`|     & ?`    & |\copyright| & \copyright \\
|!`|     & !`    & |\texttrademark| & \texttrademark\\
|\euro|  & \euro & |\textregistered| & \textregistered\\
|\dag|   & \dag  & |\ddag| & \ddag \\
\end{tabular}
\end{center}

(Note: The |\euro| command requires that the command
|\usepackage{eurosym}| be part of the document's \emph{preamble} (the
part before the |\begin{document}|). It is included as part of the
|npsreport.cls| file.)

\section{Fonts}
Like Microsoft Word, \LaTeX{} makes it easy to alter font, size, face,
and weight of text. But unlike Word, these changes are typically done
in a structured manner that lends itself to creating documents that
have consistent font usage throughout. 

\subsection{Changing Font Size}
Although \LaTeX{} allows you to use fonts of any size, the built-in
templates provides eleven built-in sizes. These sizes automatically
adjust depending on if you are creating a document with 12-point font
(NPS standard), 11-point font, or 10-point font. 

When you change the font size, that change stays in effect until you
change it again. You can confine your font change by placing the text
you want resized within braces, sometimes called a \emph{group} or a
\emph{block}, as shown in the examples in the
following table:

\begin{center}
\begin{tabular}{llll}
Size          & Point Size & Typed & Displayed  \\\hline
|\tiny|       &  6 & |{\tiny This is tiny}| & {\tiny This is tiny}\\
|\scriptsize| &  8 & |{\scriptsize scriptsize}| & {\scriptsize scriptsize} \\\
|\footnotesize| & 10 & |{\footnotesize footnotesize}| & {\footnotesize footnotesize} \\\
|\small|        & 11 & |{\small This is small}| & {\small This is small} \\\\\
|\normalsize|   & 12 & |{\normalsize normalsize}| & {\normalsize  normalsize} \\\\\\\
|\large|        & 14 & |{\large This is large}| & {\large This is large} \\\\\\\
|\Large|        & 17 & |{\Large Large}| & {\Large Large} \\\\\\\
|\LARGE|        & 20 & |{\LARGE LARGE}| & {\LARGE LARGE} \\\\\\\
|\huge|         & 25 & |{\huge huge}| & {\huge huge} \\\\\\\
|\Huge|         & 25 & |{\Huge Huge}| & {\Huge Huge} \\\\\\\
\end{tabular}
\end{center}

Notice that with the NPS template there is no difference between
|\huge| and |\Huge|.

You can also pick an arbitrary size by using the |\fontsize| and
|\selectfont| commands. The |\fontsize| command takes two arguments:
the size of the font and the size of the leading, or the amount of
space between lines. First the size is selected with the
|\fontsize{i}{j}| command where |i| and |j| are expressed in points
(there are 72.27 points in an inch). Next the font is selected with the |\selectfont| command, as shown below:

\begin{center}
\begin{tabular}{l|l}
Typed & Displayed  \\\hline
|{\fontsize{4}{5}\selectfont very tiny}|& {\fontsize{4}{5}\selectfont very tiny}\\
|{\fontsize{64}{64}\selectfont Big}|& {\fontsize{64}{64}\selectfont Big}\\
\end{tabular}
\end{center}

\subsection{Changing Font Style}
\LaTeX{} provides these macros for selecting font styles:

\begin{center}
\begin{tabular}{l|l}
Typed                                  & Displayed\\\hline
|\textrm{This is Roman}|               & \textrm{This is Roman} \\
|\textbf{This is bold}|                & \textbf{This is bold}\\
|\texttt{This is typewriter}|          & \texttt{This is typewriter}\\
|\textsc{This is small capitals}|      & \textsc{This is small capitals}\\
|\textsl{This is slanted}|             & \textsl{This is slanted}\\
|\textsf{This is sans serif}|          & \textsf{This is sans serif}\\
|\textit{Italics}|                     & \textit{Italics}\\
|\emph{This is emphasized}|            & \emph{This is emphasized}\\
|$\cal CALLIGRAPHICS$|                 & $\cal CALLIGRAPHICS$\\
|{\boldmath $\cal BOLD CALIGRAPHICS$}| & {\boldmath $\cal BOLD CALIGRAPHICS$}\\
\end{tabular}
\end{center}

Notice that the last two are surrounded by dollar signs as they require math mode (see \S\ref{math}).

If you just want to put something into italics, you should use
|\emph{text}|, which will produce \emph{text}. The reason to use
|\emph{}| and not |\textit{}| is that |\emph{}| will nest as
necessary.  For example, this:

\begin{center}
\begin{Verbatim}
\emph{You can even \emph{emphasize} a word within a sentence
that is emphasized.}
\end{Verbatim}
\end{center}

typesets as this:

\begin{center}
\emph{You can even \emph{emphasize} a word within a sentence that is emphasized.}
\end{center}

As you should in general avoid underlining text, we will not show you
how to do it in this document.

\subsection{Choosing an Arbitrary Font}
There are many ways that you can request arbitrary fonts for small
sections of your document, but they are all beyond the scope of this
article. It is also possible to embed arbitrary Unicode within a
\LaTeX{} document, either by using the Unicode-aware version of
\LaTeX{} called \raisebox{-3pt}{\includegraphics[height=12pt]{images/xelatex}} (|xelatex|), or by saving your Unicode characters in a |.pdf| file and
including that file with the |\includegraphics{}| command (as we
did with the \raisebox{-3pt}{\includegraphics[height=12pt]{images/xelatex}} logo).

If you want to change the font for an entire document, please refer to the
\LaTeX{} Font Catalogue at \url{http://www.tug.dk/FontCatalogue/}
which provides documentation and examples.  To change the font of the
NPS template, refer to \S\ref{thesisprologue}.

\section{Math}\label{math}

Typsetting mathematics is one of the primary design goals of
\LaTeX. The program  has more features for typsetting math
than typsetting text. There is also a powerful set of mathematical
extensions by the American Mathematical Society called |amsmath|. 
For more detailed information, please see:

\begin{itemize}
\item The \LaTeX{} wikibook,
  \url{http://en.wikibooks.org/wiki/LaTeX/Mathematics}.  
\item The User's guide for the \texttt{amsmath} package is available at:
\url{ftp://ftp.ams.org/ams/doc/amsmath/amsldoc.pdf}
\item The short math guide, available at
  \url{http://tinyurl.com/63w3mnu}.
\end{itemize}

What follows here is necessarily very brief. 

\subsection{Math Mode}
To typset math you must enter math mode. There are two easy ways to
enter math mode. 

\begin{itemize}
\item You can put your math between two dollar signs. For example,
  entering |$1+1=2$| in your document will produce $1+1=2$.
\item You can put the equation on a line by itself in an \emph{equation}
  environment (\eg between |\begin{equation}| and |\end{equation}|
  commands). An equation environment creates a block that is typeset in
  math mode and includes a numbered equation. For example, this:
\begin{Verbatim}
\begin{equation}
1+1=2
\end{equation}
\end{Verbatim}

produces this:

\begin{equation}
1+1=2
\end{equation}
\end{itemize}

Math mode can also be used in paragraphs to add special math characters, such as the $\pi$ symbol (using |$\pi$| here).
In fact, many of the symbols that \LaTeX{} displays can only be
displayed while in math mode.

\subsection{Simple Math in Math Mode}
As the examples above show, you can type simple math in
math mode and get what you want. In general, variables (the letters a
through z), the plus (|+|), minus (|-|) and equals (|=|) all 
typeset properly when you type them between dollar signs.  But there are some caveats:

\begin{itemize}
\item If you wish to typeset a multiplication symbol, use |\times| instead
of an asterisk (|*|).
\item If you wish to typeset division, use |\div| to enter a division
  symbol or the |\frac| command to create a fraction. Do not use |/|
  for division. 
\item Spaces are ignored in math mode. If you want a space, you
  probably should use multiple equations, with each equation
  in math mode but with a non-math mode space between them.
\end{itemize}

The table below shows some examples:

\begin{center}
\begin{tabular}{l|l|l}
Typed                   & Displayed & Comments \\\hline
|$1+2=3$|               & $1+2=3$ \\
|$10 * 10 = 100$|       & $10 * 10 = 100$  & {\small Don't use asterisks for multiplication.}\\
|$10\times10=100$|      & $10\times10=100$ & Spaces don't matter\\
|$10  \times  10  =  100$|      & $10  \times  10  =  100$\\
|$a=3$|                 & $a=3$\\
|$a=f/m$|               & $a=f/m$ & {\small Don't use the slash for division.}\\
|$a=f \div m$|          & $a=f \div m$\\
|$a=\frac{f}{m}$|       & $a=\frac{f}{m}$\\
|$f(x)=3x$|             & $f(x)=3x$\\
|$a=2b$ and $a+2=b+4$|  & \multicolumn{2}{l}{$a=2b$ and $a+2=b+4$}\\
\end{tabular}
\end{center}

Use math mode when you need to enter math---it's worth the effort. For example, consider a function of t. You can certainly type this
without math mode---witness f(t)---but doesn't it look much better
when dollar signs are placed around the symbol, like this: $f(t)$? Math
mode adds clarity.

If you wish to discuss an important equation in your document,
use the \emph{equation}
environment.  This environment sets your equation off from the body
text and gives it a number.

Using the equation environment, this:

\begin{Verbatim}
\begin{equation}
a = 1+2
\end{equation}
\end{Verbatim}

typesets as:

\begin{equation}
a = 1+2
\end{equation}



\subsection{Superscripts and subscripts}
In math mode the caret (|^|) is used for superscript and the
underbar (|_|) is used for subscript (this is why the characters are
special). The commands only superscript or subscript the following
character; if you want to superscript or subscript multiple characters
you need to make them a group by enclosing them in braces.

Here are some examples:

\begin{center}
\begin{tabular}{l|l}
Typed                   & Displayed\\\hline
|$a^2+b^2=c^2$|         & $a^2+b^2=c^2$\\
|$2^{16}=65,535$|       & $2^{16}=65,535$\\
|$N_A$ is Avogadro's constant| & $N_A$ is Avogadro's constant \\
|$A^{B^C}$|             & $A^{B^C}$\\
|$A^{B^{C^D}}$|             & $A^{B^{C^D}}$\\
|$a_k$ and $b_k$|       & $a_k$ and $b_k$         \\
|$a_k$ and $b_k$|               & $a_k$ and $b_k$ \\
\end{tabular}
\end{center}

\subsection{Combining Symbols in Groups}

Many math symbols use subscripts and superscripts to determine
placement of specific equation elements.  This includes |\int| which is used to create integrals
and |\sum| which is used to create sums. Below are some examples with the Fourier series.

\begin{center}
\begin{tabular}{l|l}
Typed                   & Displayed\\\hline
\\
\verb|$a_k=\frac{1}{\pi}|                  & $a_k = \frac{1}{\pi}\int_{-\pi}^{\pi}f(x)cos(kx)dx$\\
\verb|\int_{-\pi}^{\pi}f(x)cos(kx)dx$|      & \\[4pt]
\hline
\\
\verb|$b_k=\frac{1}{\pi}|                  & $b_k = \frac{1}{\pi}\int^{\pi}_{-\pi}f(x)\sin(kx)dx$\\
\verb|\int^{\pi}_{-\pi}f(x)\sin(kx)dx$|      & \\[4pt]
\hline
\\
\verb|$f(x)=\sum_{k=-\infty}^{\infty}| & $f(x)=\sum_{k=-\infty}^{\infty}c_k e^{-jkx}$\\
\verb|c_k e^{-jkx}$|                 & \\[4pt]
\end{tabular}
\end{center}

The |\sum| and |\int| symbols display these elements differently in an \emph{equation} environment:

\begin{equation}
f(x)=\frac{a_0}{2} + \sum_{n=1}^{\infty}\left[a_k \cos(kx) + b_k \sin(kx)\right]
\label{fourier}
\end{equation}

We used this code to type \eqnref{fourier}:
\begin{Verbatim}
\begin{equation}
f(x)=\frac{a_0}{2} + \sum_{n=1}^{\infty}
    \left[a_k \cos(kx) + b_k \sin(kx)\right]
\label{fourier}
\end{equation}
\end{Verbatim}

There are several important math mode conclusions to draw from these examples:
\begin{itemize}
  \item Simple one letter subscripts and superscripts do not need to be enclosed by curly braces, but multiple character ones must.
  \item Subscripts and superscripts in math mode are to the right of
    the sum and integral characters; however, in the equation
    environment they are above and below.  In the other uses
    (exponentials, \etc), the results are the same in either mode.
  \item The order of appearance of subscript than superscript or vice versa does not change the results.
\end{itemize}


The |amsmath| package has additional environments, symbols and
commands such as provisions for non-numbered,
multiple-lined and aligned equations. Its user guide is an
excellent reference and provides many examples.

Although math mode may seem cumbersome at first, its syntax does become second nature and very sophisticated equations can be generated, if needed.
Consult the user manuals and references provided in \S\ref{math}.

\subsection{Parenthesis and Brackets}
You can use regular parentheses in math mode, but they do not stack nicely:

\begin{center}
\begin{tabular}{l|l}
Typed                   & Displayed\\\hline
|$((1))$| & $((1))$\\
|$[((1))]$| & $[((1))]$
\end{tabular}
\end{center}


You probably want the outer brackets and parenthesis to be bigger than
the inner ones. You can do that using the |\left| and |\right|
commands.  Following these macros is another character, such as a
parenthesis or bracket.  
These macros place the correctly sized
specified character into the equation.  They will automatically get
bigger as necessary, especially in the \emph{equation} environment.

For example, here is what you get with the conventional parenthesis:
\begin{Verbatim}
\begin{equation}
A = ( \sum_{i=1}^{10}i\times\sin(i) )
\end{equation}
\end{Verbatim}

\begin{equation}
A = ( \sum_{i=1}^{10}i\times\sin(i) )
\end{equation}

And here is an example using |\left| and |\right|:

\begin{Verbatim}
\begin{equation}
A = \left( \sum_{i=1}^{10}i\times\sin(i) \right)
\end{equation}
\end{Verbatim}

\begin{equation}
A = \left( \sum_{i=1}^{10}i\times\sin(i) \right)
\end{equation}

This example  seems silly, but shows  how powerful
|\left| and |\right|  are:

\begin{Verbatim}
\begin{equation}
\left(\frac{
    \left(\frac{1}{2}\right)}{
    \left(\frac{3}{4}\right)
}\right)
\end{equation}
\end{Verbatim}

Produces this:

\begin{equation}
\left(\frac{
    \left(\frac{1}{2}\right)}{
    \left(\frac{3}{4}\right)
}\right)
\end{equation}


%%%%%%%%%%%%%%%%%%%%%%%%%%%%%%%%%%%%%%%%%%%%%%%%%%%%%%%%%%%%%%%%

\section{Spacing, Frameboxes, and Centering}
\LaTeX{} has a number of commands for controlling space, creating
boxes, and centering text.

\subsection{Controlling Spaces}

You can use these commands for controlling how much space is inserted
between words:

\begin{center}
\begin{tabular}{l|l}
Typed                                 & Displayed\\\hline
|\,| (a slash followed by a comma)    & produces\,a\,small\,space. \\
|\ | (a slash followed by a space)    & produces\ a\ standard\ word\ space. \\
|\@| (a slash followed by an at sign) & produces an intersentence space. \\
\end{tabular}
\end{center}


The small space and standard word space are used between the words in the right column.  An intersentence space is needed to correct the following error where a sentence ends in a capital letter.

\subsection{Suppressing Orphans and Widows}
When typesetting, paragraphs that have
their last line on the following page are called \emph{orphans}, and
paragraphs that begin at the bottom of a page with a single line are
called \emph{widows}. Orphans and widows are considered
ugly. Microsoft Word can be programmed to prevent widows and orphans
by requiring that all paragraphs have at least 2, 3 or 4 lines on a
page. Microsoft Word achieves this desired result by inserting spaces
between paragraphs to balance out the page as necessary. \LaTeX{} has
no automatic control over orphans and widows. It is one of the major
failings of the system.

NPS students working on their Master's thesis are frequently
instructed by the Thesis Processor to modify their document so that
there are no widows or orphans. One way to achieve such a
result is to rewrite the text by inserting sentences or removing
them. This may seem excessive to you. Another way you can control
widows and orphans is by manually adding or removing space, or by
stretching a page to allow additional lines on it.

You can force a blank line using |\\| and have the option to force a
specific length using |\\[3pt]|:
\begin{center}
\begin{tabular}{l|l}
Typed        & Displayed\\\hline
|This is\\|  & This is\\
|an example.| & an example.\\
\\
|This is\\[3pt]|  & This is\\[3pt]
|an example.| & an example.
\end{tabular}
\end{center}

You can add or remove space on a page with the |\enlargethispage| command. For
example, to squeeze another line onto the current page, insert this
command onto the page:

\begin{verbatim}
    \enlargethispage{1pc}
\end{verbatim}

This command is useful when you need to manually enlarge a
page so that the last line of a paragraph can fit on the present page
without being pushed to the next page. 

You can shorten a page by a line, forcing a widow onto the next page:

\begin{verbatim}
    \enlargethispage{-1pc}
\end{verbatim}


These commands are also useful when writing a conference paper that
needs to fit within a certain page length.

\subsection{Frameboxes and Centering}

You can draw a box around text with:

|\framebox[width]{textstring}|

The |[width]| parameter is optional. Without it, the box defaults to
the minimum size necessary to hold the \texttt{textstring}.

Here are is an example to show what we mean. This text:

\begin{Verbatim}
\framebox{This is an important statement.}\\
\framebox[15pc]{This is an important statement.}\\
\framebox[30pc]{This is an important statement.}
\end{Verbatim}

will typeset as this:

\framebox{This is an important statement.}\\
\framebox[15pc]{This is an important statement.}\\
\framebox[30pc]{This is an important statement.}

If you provide a space that is too small, the results will be ugly:

\begin{Verbatim}
\framebox[5pc]{This is an important statement.}
\end{Verbatim}

\framebox[5pc]{This is an important statement.}

You can center any text, table, figure, \etc with the \emph{center} environment, using the following:

\begin{Verbatim}
\begin{center}
\framebox{$a^2 + b^2 = c^2$}
\end{center}
\end{Verbatim}

Produces this:

\begin{center}
\framebox{$a^2 + b^2 = c^2$}
\end{center}

If you are entering a number of equations in your document, you may
want to use the \texttt{equation} environment, which will provide
numbered equations:

\begin{Verbatim}
\begin{equation}
a^2 + b^2 = c^2
\end{equation}
\end{Verbatim}
\begin{equation}
a^2 + b^2 = c^2
\end{equation}

Unlike many things in \LaTeX, the \emph{framebox} and \emph{equation}
environments cannot be readily combined. That's because |\fbox| turns
off math mode, so you need to manually turn it back on.

Type this:

\begin{Verbatim}
\begin{equation}
\fbox{$a^2 + b^2 = c^2$}
\end{equation}
\end{Verbatim}


To produce this:

\begin{equation}
\fbox{$a^2 + b^2 = c^2$}
\end{equation}



\section{Lists}

There are three kinds of lists that you may wish to make:

\begin{description}
\item[description] lists are used for definitions, where a  short phrase is bolded and the remainder text is the standard font (like this list).
 \item[enumerate] lists are lists where each item is numbered and the
  ordering is relevant, like the steps of a recipe. 
\item[itemize] lists are lists where each item is of equal
  importance.
\end{description}

Lists are implemented as \LaTeX{} environments, which means that they
begin with a |\begin{|\emph{listname}|}| and end with an
|\end{|\emph{listname}|}|.

\begin{minipage}{.45\textwidth}
\begin{Verbatim}
\begin{description}
\item[Earth] Third Planet.
\item[Mars] Fourth Planet.
\item[Venus] Second Planet.
\end{description}
\end{Verbatim}
\end{minipage}
\hfill
\begin{minipage}{.45\textwidth}
\begin{description}
\item[Earth] Third Planet.
\item[Mars] Fourth Planet.
\item[Venus] Second Planet.
\end{description}
\end{minipage}


\begin{minipage}{.45\textwidth}
\begin{Verbatim}
\begin{enumerate}
\item Wake up.
\item Go to work.
\item Go home.
\item Go to sleep.
\item Repeat.
\end{enumerate}
\end{Verbatim}
\end{minipage}
\hfill
\begin{minipage}{.45\textwidth}
\begin{enumerate}
\item Wake up.
\item Go to work.
\item Go home.
\item Go to sleep.
\item Repeat.
\end{enumerate}
\end{minipage}

\begin{minipage}{.45\textwidth}
\begin{Verbatim}
\begin{itemize}
\item Hamburgers
\item Hotdogs
\item Chips
\end{itemize}
\end{Verbatim}
\end{minipage}
\hfill
\begin{minipage}{.45\textwidth}
\begin{itemize}
\item Hamburgers
\item Hotdogs
\item Chips
\end{itemize}
\end{minipage}


\section{Labels and Captions}\label{sec:labels}
Labels are hidden markers in your |.tex| files created by the
|\label{name}| command.  These markers are never shown directly in the
output files.  However, correctly placing these markers in your file
allows you to reference chapters, appendices, sections, figures,
tables and equations.  You may wish to give your marker names prefixes
such as \texttt{chap:}, \texttt{sec:}, \texttt{fig:}, \texttt{tab:}
and \texttt{eqn:} to logically identify the labels.

Captions are the text that appears below a figure or table to provide
context for the information presented.  Captions are indicated by the
command \verb|\caption{Sentence.}| Sometimes longer captions can look
poorly in the Lists of Figures and Tables, so a caption for the table
can also be specified by using
\verb|\caption[Short Sentence.]{Longer Sentence.}|

\section{Tables}
Tables provide a valuable means to display data in an organized
manner.  Using tables in \LaTeX{} is easy once the syntax is
understood, although tables can be incredibly complex as well.  To see
plenty of examples and explanation of options, visit
\url{http://en.wikibooks.org/wiki/LaTeX/Tables}. There are also
\emph{many} add-on packages which provide additional functionality.

Here is a very simple table:

\begin{Verbatim}
\begin{center}
\begin{tabular}{cc}
Shape & Sides \\
\hline
Triangle & 3 \\
Square & 4 \\
\end{tabular}
\end{center}
\end{Verbatim}

Produces this:

\begin{center}
\begin{tabular}{cc}
Shape & Sides \\
\hline
Triangle & 3 \\
Square & 4 \\
\end{tabular}
\end{center}

This table has six things we have not seen before:

\begin{enumerate}
\item The |{cc}| is the table format specifier. Each ``c'' represents
  a column that is centered.
\item The |&| is used to separate columns of the table.
\item The |\\| is used to separate rows of the table.
\item The |\hline| is used to draw a horizontal line.
\item The table is drawn in a |tabular| environment.
\item The |center| environment is used to center the table.
\end{enumerate}

Here is a slightly more complex table:

\begin{Verbatim}
\begin{center}
\begin{tabular}{lcr}
Left Justified & Center Justified & Right Justified \\
\hline
Each & Column & Entry \\
Spaces      & Do not    & Matter \\
But         & May       & Assist you. \\
\end{tabular}
\end{center}
\end{Verbatim}

And here is how it formats:

\begin{center}
\begin{tabular}{|lcr}
Left Justified & Center Justified & Right Justified \\
\hline
Each & Column & Entry \\
Spaces      & Do not    & Matter \\
But         & May       & Assist you. \\
\end{tabular}
\end{center}

This table has three columns as identified by the format specification
|lcr|. The first column is left-justified (``l''), the middle column
is center justified (``c'') and the third column is right justified
(``r''). The left side of the table has a vertical bar due to the pipe character (\verb+|+).  The |\hline| macro causes a horizontal line to be drawn across the table. 
Columns are indicated by the ampersands (|&|) and the amount of spaces used are not important.  Thus, the spaces can be used to ensure your columns line up in your |.tex| file the way the table is intended to print.
An empty cell is just ampersands separated by a space.
To start a new row of the table, use the double backslash (|\\|); this can be done at the end of the current row.

Below is a more complex table:


\begin{Verbatim}
\begin{center}
\begin{table}
\begin{tabular}[t]{|l||r|r|r|r|r|r|r|}
\hline
Pulse  & Pulse  & PRF & Duty  & Coupler & Power & Average & Peak  \\
Width  & Length &     & Cycle & Losses  & Meter & Power   & Power \\
\hline
Long & $1.0 * 10^{-6} s $ & 750 Hz & $7.0 * 10^{-4} $ & 
       49.1 dB & 3.4 dBm & 52.5 dBm & 73 dBm \\
& & & 31.075 dB & & & & 43 dBW \\
& & & & & & & 227.7 kW \\
\hline
Medium & $3.0 * 10^{-7} s $ & 1200 Hz & $2.0 * 10^{-4} $ & 
         49.1 dB & 0 dBm & 49.1 dBm & 83 dBm \\
& & & 34.29 dB & & & & 53 dBW \\
& & & & & & & 218.27 kW \\
\hline
Short & $1.0 * 10^{-7} s $ & 2400 Hz & $1.9 * 10^{-4} $ & 
        49.1 dB & -1 dBm & 48.1 dBm & 83 dBm \\
& & & 35.05 dB & & & & 53 dBW \\
& & & & & & & 206 kW \\
\hline
\end{tabular}
\end{center}
\end{Verbatim}

And here is the formatted table:

\begin{center}
\begin{tabular}[t]{|l||r|r|r|r|r|r|r|}
\hline
Pulse  & Pulse  & PRF & Duty  & Coupler & Power & Average & Peak  \\
Width  & Length &     & Cycle & Losses  & Meter & Power   & Power \\
\hline
Long & $1.0 * 10^{-6} s $ & 750 Hz & $7.0 * 10^{-4} $ &
       49.1 dB & 3.4 dBm & 52.5 dBm & 73 dBm \\
& & & 31.075 dB & & & & 43 dBW \\
& & & & & & & 227.7 kW \\
\hline
Medium & $3.0 * 10^{-7} s $ & 1200 Hz & $2.0 * 10^{-4} $ &
         49.1 dB & 0 dBm & 49.1 dBm & 83 dBm \\
& & & 34.29 dB & & & & 53 dBW \\
& & & & & & & 218.27 kW \\
\hline
Short & $1.0 * 10^{-7} s $ & 2400 Hz & $1.9 * 10^{-4} $ & 
        49.1 dB & -1 dBm & 48.1 dBm & 83 dBm \\
& & & 35.05 dB & & & & 53 dBW \\
& & & & & & & 206 kW \\
\hline
\end{tabular}
\end{center}

\section{Graphics}
This section briefly describes how to embed graphics in a \LaTeX{} document. For alternative
treatments we recommend H\"oppner's  ``Strategies for including graphics in \LaTeX{} documents''\cite{strategies}, and the book \emph{The LaTeX Graphics Companion}\cite{graphics-companion}.

Graphics are embedded using the |\includegraphics| command. It looks like this:

\begin{Verbatim}
       \includegraphics[options]{filename}
\end{Verbatim}

Typical options that you can use are |width=XX| or |height=XX|. For example, to include an image, have it centered, and scale it to 3 inches, use this:

\begin{Verbatim}
\begin{figure}
    \begin{center}
	\includegraphics[width=3in]{imagedirectory/file}
	\caption{Caption of the important figure.}
	\label{fig:importantfigure}
    \end{center}
\end{figure}
\end{Verbatim}

The |imagedirectory/file| notation indicates the images are in a subdirectory of |imagedirectory| and the file name is |file|.  The file extensions (|.jpg|, \etc) are optional. 

\emph{Note: \LaTeX{} does not properly include graphics that have
  periods in their filenames in addition to the period that is used to
  denote the file type!} 

There are two kinds graphics that you can include in a \LaTeX{}
document: 
\begin{description}
\item[Vector graphics] are excellent for printing purposes.  These images
 show up nicely on the computer screen and paper.  The computer renders these dynamically.  Several tools to generate vector graphics include |Inkscape| and
 \texttt{asymptote}. In general, you should convert vector graphics to a |.pdf| file and embed it in your document using |\includegraphics|.
\item[Raster graphics] include |.bmp|, |.jpg|, |.png|, and |.gif| images.  Formats without compression, including |.bmp| files
 generally will make your thesis file much larger.  Some of these images show up poorly or 
 pixelated in print copy.  You should convert these files to |.jpg| or |.png| and embed them in your document with |\includegraphics|.
\end{description}

Grahpics can also be created with R, gnuplot, matplotlib, or the
asymptote package.

Specific macros for including graphics provided by the |npsreport|
template are discussed in \secref{graphics}.

\section{Bib\TeX{} and Citations}
Bibliography and citation are important in your thesis.  Each department has different expectations
on citation styles.  Even if you select the wrong style initially, changing the format to the correct
option is only a one-line change for \LaTeX{}---unlike a complete nightmare for Microsoft Word users.

See \url{http://en.wikibooks.org/wiki/LaTeX/Bibliography_Management} and
 \\ \url{http://stefaanlippens.net/bibentry} for more information on BiB\TeX.
 
Many common citation reference organizers and websites output Bib\TeX{} formatted citations in |.bib| files.  This is a time saver,
as rather than retyping information, you can download the |.bib| file and copy its contents into your thesis master |.bib| file.
The master |.bib| file contains all of your citations, even ones that you have not yet cited.  When you reference these citations in your thesis, Bib\TeX{} can 
generate the references list for you.  To cite a specific reference use the \verb|\cite{name}| command.  After your first run of |pdflatex|, you follow with the |bibtex| command 
as in \S\ref{runninglatex}. 

\section{Going Further}
You are now on your way to becoming a \LaTeX{} expert and will find that many of the \LaTeX{} modules are already installed with your \LaTeX{}. 
You may find the following packages useful:

\begin{description}
\item[multirow] Allows a single table cell to extend to multiple
rows.

\item[ifthen]  Allows you to put conditions in your thesis. It is a
bit easier than using the |if| that's built in to \TeX.

\item[acronym]  A great package for automatically generating acronym lists.
It can track the usage of acronyms to correctly use the long name of the acronym
on its first use.

\item[makeidx]  For creating an index of key words and phrases in your document and what page the keywords appear on.

\item[mcode]  Places a MATLAB .m file contents directly into the output.  The
package performs syntax highlighting and line numbering if desired.  This package is not in the CTAN
and documentation may be obtained from the MATLAB Central website at \url{http://tinyurl.com/3wgcufr}.

\end{description}

In learning about \LaTeX{}, you are likely to do searches on the Internet, learn
about a new package, and want to learn more about it. You will
probably be surprised to find that many of these packages are already
part of your \LaTeX{} distribution. In most cases, the documentation for
the packages is already on your hard drive as a |.pdf| file.  We
recommend reading package documentation; the documentation frequently
has better information than the random pages you may find by using Google.

If, however, you do need to download a \LaTeX{} style or package file manually, 
the easiest way to include a |.sty|, |.cls| or similar file is to place it in the same folder as your current document.  If you choose to place it in your LaTeX installation folder (perhaps to use on many documents), a manual installation step must be performed for your \LaTeX{} distribution:

MiKTeX users perform:
\begin{enumerate}
  \item |Click Start - Programs - MiKTeX 2.9 - | \\ 
    |Maintainance (Admin) - Settings (Admin)|
  \item After the dialog opens, click the |Refresh FNDB| button.
\end{enumerate}

Other distributions (MacTeX, TeX Live, etc), should try |mktexlsr| or |texhash| on the command prompt (you may need to run this command as the root user).


\chapter{The NPS \LaTeX{} Template Package}
This chapter describes how to get the NPS report template and how to
use it.

\section{Getting the Template}
Get a copy of the \texttt{npsthesis.zip} distribution from
\url{http://simson.net/npsthesis/npsthesis.zip}. Unpack this into a
directory on your computer. This is where we will be working for the
remainder of this chapter.

\subsection{\LaTeX{} Files Included in the Template Package}

Below you will find the important files in the package.

These files are used for all document types:

\begin{description}
\item[Makefile] The Makefile to make the thesis

\item[appendix1.tex] The example file for an appendix.

\item[authorindex.*] The \LaTeX~authorindex package, for making
the Referenced Authors page.

\item[chapter1.tex] The example file for each chapter.

\item[chngcntr.sty] The \texttt{chngcntr} package, for changing
the way that \LaTeX~displays its counters.

\item[fixerrors.py] A python program that removes the breaks in the
\texttt{.bbl} file inserted by Bib\TeX~and improper authorindex items.

\item[npsreport.cls] The style class file for NPS documents.

\item[nps-plain.bst] A Bib\TeX~style file that makes references in a style that is acceptable to NPS for which the references appear sorted by author's last name.

\item[nps-plain-unsorted.bst] A Bib\TeX~style file that makes
  references in a style that is acceptable to NPS for which the
  references appear in  the order of appearance.

\item[nps-plain-classified.bst] A Bib\TeX~style file that makes references in a style that is acceptable to NPS for a classified thesis. References are sorted by last name.

\item[nps-plain-classified-unsorted.bst] A Bib\TeX~style file that makes references in a style that is acceptable to NPS for a classified thesis. References appear in the order of appearance.

\item[nps\_logo\_3clr\_cymk.pdf] NPS Logo, 3 color, in format suitable for printing

\item[thesis.tex] A skeletal thesis \LaTeX{} template file

\item[thesis.bib] A skeletal thesis bibliography file
\end{description}

These files are skeletal files for creating your own documents. Use
them as a template by removing our text and inserting your own:

\begin{description}
\item[thesis.tex] A one-author thesis.
\item[thesis\_two.tex] A two-author thesis.
\item[thesis\_coadvisors.tex] A one-author thesis with two
  co-advisors.
\item[thesis.bib] A thesis BiB\TeX{} input file.
\end{description}

You will also find |techreport.tex|, which is the \LaTeX{} source code for this document.

\subsection{\LaTeX{} Demonstration Files}

In addition to the files in the |npsthesis.zip| file, we have made
available a set of demonstration documents. These can be downloaded
from \url{http://simson.net/npsthesis/demos.zip} and includes the following files:

\begin{description}
\item[demo\_classified.tex] A demonstration classified master's thesis that
  shows how to use all of the macros we have created for labeling
  classified paragraphs, figures and references. To avoid confusion,
  this document is classified F//MM//SPECIAL//TOM FOOLERY (F is for Fun).
\item[demo\_fouo.tex] A demonstration For Official Use Only
  thesis. To avoid confusion this document is classified For
  Entertainment Use Only (FEUO).
\item[demo\_phd.tex] A demonstration PhD thesis.
\item[demo\_report.tex] A demonstration technical report.
\item[demo\_thesis.tex] A demonstration master's thesis.
\item[demo\_traditional.tex] A demonstration thesis in the traditional
  NPS master's thesis style.
\item[demo\_twoauthor.tex] A demonstration master's thesis with two authors.
\end{description}


\section{Creating Your Document}
The skeletal |thesis.tex| file consists of two main parts: the
\emph{prologue} (everything before the |\begin{document}|) and the
\emph{body} (everything between the |\begin{document}| and the
|\end{document}|). The body is further split into two parts: the main
body and the postmatter (the appendices, bibliography, and
distribution list). You will typically create your thesis or technical
report by editing
each. Some students put their entire thesis into the |thesis.tex|
file, while others put each chapter into its own |.tex| file and
include them using the \verb|\include{filename.tex}| command.

The remainder of this section will show a skeletal thesis template for
each of these three parts (the prologue, the main body and the
postmatter), and then will explain the purpose of each command.

\subsection{The Thesis Prologue}\label{thesisprologue}
Below is the thesis prologue from the |thesis.tex| file, with all of
the comments removed:
\begin{Verbatim}[fontsize=\small]
\documentclass[twoside,thesis,authorindex]{npsreport} 
\securitybanner{}
\title{[TITLE]}
\author{[AUTHOR]}
\degree{Master of Science in [DEGREE]}
\degreeabbreviation{MS}
\department{Department of [DEPARTMENT]}
\thesisadvisor{[ADVISOR]}
\secondreader{[SECOND READER]}
\departmentchair{[DEPARTMENT CHAIR]}
\rank{[RANK]}
\prevdegrees{[UNDERGRADUATE DEGREE]}  
\degreedate{[DEGREE DATE]} 
\distribution{Approved for public release; distribution is unlimited}
\abstract{
  [INSERT ABSTRACT HERE]
}
\ReportType{Master's Thesis}
\DatesCovered{2102-06-01---2104-10-31}
\SponsoringAgency{Department of the Navy}
\RPTpreparedFor{}
\ReportClassification{Unclassified}
\AbstractClassification{Unclassified}
\PageClassification{Unclassified}
\SupplementaryNotes{ The views expressed in this thesis are those of
  the author and do not reflect the official policy or position of the
  Department of Defense or the U.S. Government.
  \footnotesize IRB Protocol Number: XXXX}
\SignatureOne{\includegraphics[width=2in]{signature_picture}}
\makeatletter\@removefromreset{footnote}{chapter}\makeatother
\end{Verbatim}

The following explains each of these commands and options:

\begin{description}
  \item[$\backslash$documentclass] \hfill \\
  The documentclass specifies that the document uses the |npsreport.cls|
  file and all settings contained therein.  There are  several optional
  parameters, each separated by comma: 
  \begin{description}
    \item[article, thesis, or dissertation] choose the appropriate one 
    for the case.
    \item[12pt, 11pt, or 10pt] Font size selection.  With no option given, 12pt is the default.
    \item[times, arial, or courier]  Font selection.  With no option given, times is the default.
    \item[twoauthors, threeauthors, or fourauthors] use these options if you have 
    several authors.  Single authors need no option. 
    \item[twoadvisors] if you have two advisors rather than a second reader. 
    \item[twoside] prints on both sides of the same sheet of paper; recommended.
    \item[classified] if you are using an approved computer system to 
    write your thesis on sensitive research.
    \item[authorindex] if you are including an author index page of your 
    thesis references.
    \item[index] if you are including a keyword index page of your 
    thesis important terms.
    \item[acronym] for a more sophisticated handling of acronyms.  See 
    |acronyms.tex| for additional information.
    \item[traditional] prints the thesis in the style of the NPS
      Microsoft Word thesis template. Although you are free to use
      this style, the newer style is approved and looks quite nice 
    when no option is given.
    \item[singlespace] if you prefer single-spaced paragraphs, though it 
    may be a little harder to read. This is not approved for an NPS
    Masters Thesis, but is approved for NPS technical reports.
    \item[tight] Causes the spacing between paragraphs and paragraph 
    indentation to be smaller than standard.
  \end{description}
  \item[$\backslash$securitybanner] \hfill \\
  Leave blank unless producing a FOUO or classified theses.  Whatever text appears 
  between the braces is placed
  at the top and bottom of each page of the document.
  \item[$\backslash$title]  \hfill \\ Your title 
  \item[$\backslash$degree] \hfill \\  Your planned NPS degree written out.
  \item[$\backslash$degreeabbreviation] \hfill \\   MA, MS, MBA, or other shorthand notation
  \item[$\backslash$prevdegrees] \hfill \\   Written out as ``B.S., Degree, School, Year'' 
  \item[$\backslash$degreedate] \hfill \\   Written as ``Month Year'' 
  \item[$\backslash$distribution] \hfill \\
  One of the approved Department of Defense distribution statements 
  (A through F or Export Control).  These are
  listed out on the thesis release form that must also be submitted 
  with your thesis.
  \item[$\backslash$abstract] \hfill \\
  Your entire abstract goes here.  
  Do not make it too big, as it must also fit on the SF298 form.
  \item[$\backslash$SponsoringAgency] \hfill \\
  Your appropriate military department, such as Department of the 
  Air Force, Department of the Navy, \etc
  \item[$\backslash$RPTpreparedFor] \hfill \\
  This optional item can be used to specify the sponsor of the research.
  \item[$\backslash$SupplementaryNotes] \hfill \\
  If your thesis does not have an Institutional Review Board (IRB) protocol 
  number, replace the XXXX with N/A, otherwise fill in 
  the appropriate number.  
  This is needed for theses that use human subjects to collect data.  
  Ask your advisor for more information if this applies. 
  \item[$\backslash$SignatureOne, SignatureTwo, SignatureThree, and SignatureFour] \hfill \\
  Each author's signature line can show an image of the signature, if desired.
  Specifying the width as 2 inches is recommended.  This is an 
  optional feature. 
\end{description}

\subsection{The Thesis Main Body}
Below is a thesis body, with all of the comments that appear on
lines by themselves removed:
\begin{Verbatim}[fontsize=\small]
\begin{document}
\NPScover                       % Cover
\NPSsftne                       % SF298
\NPSthesistitle                 % Title page
\NPSabstractpage                % Abstract Page
\NPSfrontmatter                 % NPS front matter follows
\renewcommand{\chaptermark}[1]{
  \markboth{\MakeUppercase{\chaptername}\ \thechapter.\ #1}{}}
\NPStableOfContents
\NPSlistOfFigures
\NPSlistOfTables
\NPSlistOfAcronyms{
 \begin{description}
   \item[NPS] Naval Postgraduate School
   \item[USG] United States Government
 \end{description}
}
\NPSlistOfAcronymsFromFile{acronyms}
\NPSexecsummary{
  [EXECUTIVE SUMMARY CONTENTS]
}
\NPSacknowledgements{
  [ACKNOWLEDGEMENTS CONTENTS]
}
\NPSbody
\chapter{[CHAPTER ONE TITLE]}
[CHAPTER BODY]

This is the beginning of your thesis. Don't be a Micky
Mouse\cite{mm2}: Always have text between every head and subhead.
\section{Your First Section}
[Section One Body]
\section{Your Second Section}
[Section Two Body]
\section{Your Third Section}
[Section Three Body]
\chapter{[CHAPTER TWO TITLE]}
[CHAPTER BODY]
This is the beginning of the second chapter. 
Always have text between every head and subhead.
\section{Your First Section}
[Chapter two Section One Body]
\section{Your Second Section}
[Chapter two Section Two Body]
\section{Your Third Section}
[Chapter two Section Three Body]
\end{Verbatim}

Now we describe each command:

\begin{description}
  \item[$\backslash$NPScover] \hfill \\
  Prints the coversheet page.
  \item[$\backslash$NPSsftne] \hfill \\
  Prints the Standard Form 298 completely filled out with the provided information.
  \item[$\backslash$NPSthesistitle] \hfill \\
  Prints the signature page.
  \item[$\backslash$NPSabstractpage] \hfill \\
  Prints the abstract page.
  \item[$\backslash$NPSfrontmatter] \hfill \\
  Applies some thesis settings for the remainder of the document.
  \item[$\backslash$NPStableOfContents] \hfill \\
  Creates the Table of Contents that lists |chapters| and |subsections|.
  \item[$\backslash$NPSlistOfFigures and NPSlistOfTables] \hfill \\
  These lists are automatically created based on the content of the thesis, using the \emph{figure} and \emph{table}
  environments.  
  \item[$\backslash$NPSlistOfAcronyms] \hfill \\
  Manual list of acronyms, useful for a very short list of acronyms. 
  Use this or NPSlistOfAcronymsFromFile but not both.
  \item[$\backslash$NPSlistOfAcronymsFromFile] \hfill \\
  Specifies the file of where the acronyms are stored, |acronyms.tex| in 
  this instance.  Using this separate file can keep your |thesis.tex| 
  easier to read.  Use this or NPSlistOfAcronyms but not both.
  \item[$\backslash$NPSexecsummary] \hfill \\
  Used by the Electrical Engineering, Systems Engineering, and Operations 
  Research departments.
  \item[$\backslash$NPSacknowledgements] \hfill \\
  It is considered good form at NPS to formally thank your advisor as
  well as others at NPS who have contributed in a positive manner to
  your time at the Institution. You are also free to thank family
  members, friends, team members, family pets, or anyone else you deem
  appropriate.
  \item[$\backslash$NPSbody] \hfill \\
  Thesis chapters follow.  
\end{description}

\subsection{The Postmatter}
The end of the document optionally has one or more appendices and a distribution list:
\begin{Verbatim}[fontsize=\small]
\def\showURL{}
\bibliographystyle{nps-plain-unsorted}
\bibliography{thesis}
\NPSappendixTOC{Appendix TITLE}
[APPENDIX BODY]
\NPSend         
\chapter*{Initial Distribution List}
\addcontentsline{toc}{chapter}{Initial Distribution List}
\singlespace
\begin{enumerate}
\item Defense Technical Information Center\\
  Ft. Belvoir, Virginia
\item Dudly Knox Library\\Naval Postgraduate School\\
  Monterey, California
\item Marine Corps Representative\\Naval Postgraduate School\\
  Monterey, California
\item Directory, Training and Education, MCCDC, Code C46\\
  Quantico, Virginia
\item Marine Corps Tactical System Support Activity 
  (Attn: Operations Officer)\\Camp Pendleton, California
\end{enumerate}
\end{document}
\end{Verbatim}

Now we describe these commands:

\begin{description}
  \item[$\backslash$bibliographystyle] \hfill \\ Can be one of the
    provided styles (|nps-plain|, |nps-plain-classified|,\\
    |nps-plain-classified-unsorted|, |nps-plain-unsorted|) or others
    commonly used (|acm|, |acmtrans|, |amsalpha|, |amsplain|,
    |apa-good|, |ieeetr|, |ieeetrans|, etc.)
  \item[$\backslash$bibliography] \hfill \\
  Specifies your master |.bib| file, in this case, |thesis.bib|.  All cited references should be kept in this file.
  \item[$\backslash$NPSappendix] \hfill \\
  Use this for a single appendix thesis with an ``Appendix'' entry in the Table of Contents.
  Add a \verb|\chapter{title}| creates a lettered appendix ``A.'' 
  \item[$\backslash$NPSappendixTOC\{Appen TITLE\}] \hfill \\
  Use this for a single appendix thesis with a single entry in the Table of Contents of ``Appendix: Appen TITLE.''
  The appendix is not given an appendix letter.  This is the preferred style for NPS single-appendix theses.
  Additionally, use \verb|\section*{name}| rather than \verb|\section{name}| to keep entries out of the Table of Contents.
  \item[$\backslash$NPSappendices] \hfill \\
  Use this for a multiple appendices thesis.  Each appendix will need a \verb|\chapter{title}|.
  \item[$\backslash$NPSend] \hfill \\
  Includes the authorindex and index, if the option was specified in the |documentclass|.  Concludes the content of the thesis.
\end{description}

\section{Additional Commands Provided by the Template}
In additional to commands above, the NPS template provides additional
commands designed to make it easier to have references, tables,
figures, and embedded graphics.
\subsection{Labels}\label{refcommands}
Recall from \secref{sec:labels} that labels are hidden markers in your
|.tex| files created by |\label{name}|.  The NPS \LaTeX{} template contains a number of commands for
referencing labels in your text; they are presented below:

\begin{tabular}{lp{5in}}
\multicolumn{2}{l}{Built in to \LaTeX:}\\
|\ref{l}|     & General reference of the label that places the label's number in the document. \\  
\\
\multicolumn{2}{l}{Provided by \texttt{npsreport.cls}:}\\
|\chapref{l}| & Chapter reference that formats as ``Chapter 3'' \\  
|\chapvref{l}|& Chapter reference that formats as ``Chapter 3 on page 4'' \\  
|\secref{l}|  & Section reference that formats as ``Section 3.'' You can use this for sections, subsections, and so on. \\  
|\secvref{l}| & Section reference that formats as ``Section 3 on page 4'' \\  
|\figref{l}|  & Figure reference that formats as ``Figure 3'' \\  
|\figvref{l}| & Figure reference that formats as ``Figure 3 on page 4'' \\  
|\tabref{l}|  & Table reference that formats as ``Table 3'' \\  
|\tabvref{l}| & Table reference that formats as ``Table 3 on page 4'' \\  
|\eqnref{l}|  & Equation reference that formats as ``Equation (3.1)'' \\  
|\eqnvref{l}| & Equation reference that formats as ``Equation (3.1) on page 4'' \\  
|\eqnsref{l,m}| & Equation reference that formats as ``Equations (3.1) and (3.5)'' \\  
|\eqnsvref{l,m}| & Equation reference that formats as ``Equations (3.1) and (3.5) on page 4'' \\  
|\appref{l}|  & Appendix reference that formats as ``Appendix 3'' \\  
|\appvref{l}| & Appendix reference that formats as ``Appendix 3 on page 4'' \\  
\end{tabular}

The |vref| commands can also automatically swap ``on page 4'' for ``on the preceeding page'' and other phrases.

\subsection{Tables and Figures}
Tables and figures are floating objects that \LaTeX{} moves around as
necessary to make your thesis look better. Tables are inserted with
the \verb|\begin{table}| command while figures are inserted with
\verb|\begin{figure}|. Here are some rules to consider:

\begin{itemize}
\item Every table and figure should have a caption, created with the
  |\caption{text}| command.
\item Every table and figure should have a unique label, created with
  the |\label{marker}| command.
\item Every table and figure should be referred to in the main body of
  your text. \LaTeX{} provides a command called |\ref{marker}|;
  this template provides additional commands |\tabref{marker}|
  and |\figref{marker}|. All of the reference commands are shown
  in \secvref{refcommands}.
\item Do not assume that figures and tables will be on the same text as your
  page. Always refer to the figures and tables by their numbering.
\end{itemize}


\subsection{Including Photos and Figures}\label{graphics}
The NPS report template uses the \LaTeX{} |graphicx| package to embed
photos and other graphics into the resulting document. You can include
graphics directly with the |\includegraphics| command or use the
commands described in this section.

By using the |\sgraphic{filename}{caption}| command provided by
|npsreport.cls|, you can embed a
photo from a given filename and give it a label and a caption. The
label is set to be the filename. Use the
|\figref{tag}| command to get an in-paragraph 
reference. \figref{images/home_topimg} shows an example of an
embedded image using \verb+\sgraphic+. The filename is |demos/demo_cart/home_topimg|. It is embedded
with the command:

\begin{Verbatim}[fontsize=\small]
\sgraphic{images/home_topimg}{Banner from the top of the NPS web site.}
\end{Verbatim}

The figure can then be referenced with the command:

\begin{Verbatim}
\figref{images/home_topimg} shows an example 
of an embedded image using \verb+\sgraphic+.
\end{Verbatim}

\sgraphic{images/home_topimg}{Banner from the top of the NPS web site.}

The variants of sgraphic are |b| for box, |n| for no box, |o| for boxed but not a figure, and |on| for no box and not a figure.

|\sgraphicb{file}{caption}|
\sgraphicb[width=3in]{images/photo3}{Using sgraphicb (box)}

|\sgraphicn{file}{caption}|
\sgraphicn[width=3in]{images/photo4}{Using sgraphicn (no box)}

%|\sgraphico{file}{caption}|
%\sgraphico[width=3in]{images/photo1}{Using sgraphico (no figure)}

%|\sgraphicon{file}{caption}|
%\sgraphicon[width=3in]{images/photo2}{Using sgraphicon (no figure, no box)}

Each of the \emph{sgraphic} commands have an optional parameter that
you can use to modify the image.  The |width| can be used to specify a
dimension on the page such as 3 inches or 10 centimeters.  The |scale|
can be used with either a number between 0 and 1 to scale down the
image or larger than 1 to magnify the image; magnification of
bitmapped images may look pixelated and print poorly if you are not
starting with an image that has sufficient resolution. If
you need a larger image, you should find a way to make it larger
before including it in your thesis.  The image can be rotated with
|angle|.

\begin{Verbatim}
\sgraphic[width=3in]{imagefile}{caption}
\sgraphicb[scale=0.5]{imagefile}{caption}
\sgraphicn[angle=270]{imagefile}{caption}
\sgraphico[width=2in]{imagefile}{caption}
\sgraphicon[width=10cm]{imagefile}{caption}
\end{Verbatim}

The |twofigures| command allows you to have two figures side-by-side, 
as shown in \figref{images/photo5} and \figref{images/photo6}.
An example of the |width1| and |width2| entries is 2.5in, 10cm, etc.

\twofigures{2.5in}{images/photo5}{Using twofigures}
           {2.5in}{images/photo6}{A second caption.}

\begin{Verbatim}
\twofigures{width1}{imagefile1}{caption1}
           {width2}{imagefile2}{caption2}
\end{Verbatim}

|\twoimages{imagefile1}{imagefile2}{caption}|
\twoimages{images/photo7}{images/photo8}{Using twoimages}

There are three additional commands to arrange text or images side by side.  These are 
advanced features.  See |npsreport.cls| for additional information.

\begin{Verbatim}
\sidebyside{contents1}{contents2}{caption}{label}|
\tsidebyside{contents1}{contents2}{caption}{label}|
\threesidebyside{contents1}{contents2}{contents3}{caption}{label}
\end{Verbatim}

\section{Ph.D. Dissertations}

The NPS \LaTeX{} template is also created for Ph.D. dissertations when using the |dissertation| option on the document class.  The signature page is very different from a Master's thesis.  Additional macros are available to make creating the signature page.  The following macros would all be placed into your primary |.tex| document before the |\begin{document}| (as in the prologue of \secref{thesisprologue}).

\begin{Verbatim}
\advisorOne{[Person 1]}
\advisorTwo{[Person 2]}
\advisorThree{[Person 3]}
\advisorFour{[Person 4]}
\advisorFive{[Person 5]}
\advisorSix{[Person 6]}
\end{Verbatim}

Most Ph.D. committees have no more than 6 members.  If your committee has more than this, you will still have to manually edit this signature page and expect difficulty in making everything fit on a single page.

The names of your committee members should be placed in these macros.  Usually your primary advisor is in the One entry. You should still provide the |\thesisadvisor{[ADVISOR]}| macro as well for this person.  Your committee may have guidance on the preferred order of appearance of the remaining members.

If your committee has 5 members, then use:
\begin{Verbatim}
\NPSdissertationfivememberstrue
\end{Verbatim}

as this will mute the sixth member.

Each advisor can have up to four lines for the title (this does not include their name).  However, to provide maximum control over the exact placement over each title, the lines are called out separately.

\begin{Verbatim}
\advisorOneLineOne{Dissertation Advisor}
\advisorOneLineTwo{Professor, Department of}
\advisorOneLineThree{Computer Science}
\advisorOneLineFour{}
\end{Verbatim}

When a line is to be empty, the macro does not need to be explicitly called out as in the above example; it was only provided for clarity.  The above lines will wrap the text if the entry is made too long.  This is undesirable in titles and that is why each line is called out specifically.

The other macros exist for |advisorTwo|, |advisorThree|, |advisorFour|, |advisorFive| and |advisorSix|.

Two additional signatures are needed for a Ph.D. signature page.  

\begin{Verbatim}
\assocprovost{[Associate Provost]}
\departmentchair{[Department Chairman]}
\end{Verbatim}

You should speak to your primary advisor if you do not know the appropriate names to provide.

Ideally, the signature page will be completed with normal size 12 font.  However, it is possible to have some excessively long titles that will not properly fit.  To provide for this circumstance, the macros

\begin{Verbatim}
\NPSsignaturefontsizesmall
\NPSsignaturefontsizefootnote
\end{Verbatim}

give the option of size 11 and size 10 fonts, respectively.  This font option only applies to the signature page font.

\section{Macros for Creating Classified Documents}

The NPS \LaTeX{} template has been designed so that it can be used
for creating documents that are For Official Use Only (FOUO) or
classified at any classification level. As a general rule, you only
create a classified document on a system that has been approved for
processing classified data at a particular classification level. You
should also arrange for the NPS security office to install \LaTeX{} on
the system, rather than installing it yourself. However, once you have
a system that is appropriately set up, the template can save a
substantial amount of time over the alternative.

In general, preparing a classified document requires a few
changes from preparing an unclassified document:

\begin{enumerate}
\item The security banner must be set.
\item The SF-298 form must be properly labeled.
\item Each paragraph and caption must be labeled.
\item Citations must be appropriately classified.
\end{enumerate}

\subsection{Setting the Security Banner}

The security banner is the notation that is printed at the top and
bottom of each page of your classified or otherwise restricted
document. Use the |\securitybanner{}| macro to set the banner. Here is
an example from our fictitiously classified document:

\begin{Verbatim}
    \securitybanner{F//MM//SPECIAL//TOM FOOLERY}
\end{Verbatim}

This document is classified F (for FUN) and it contains three
additional restrictions: MM (Mickey Mouse), SPECIAL, and TOM FOOLERY.  

\subsection{Labeling the SF-298}

The NPS \LaTeX{} template automatically creates a SF-298 form for
you. When you create a classified document you need to determine the
classification of the document's SF-298 form, its abstract, and the
report itself. In order for the SF-298 to be unclassified the
document's title and abstract must be unclassified. However it is
possible to have a classified document with an unclassified abstract
and an unclassified title. In this case the SF-298 may also be
unclassified. However, before you make a determination, you may wish
to speak with your sponsor or with the Site Security Officer. 

The report classification, abstract classification, and classification
of the SF-298 are indicated with these three macros:

\begin{Verbatim}
    \ReportClassification{Fun}
    \AbstractClassification{Jolly}
    \PageClassification{Amusing}
\end{Verbatim}

In this case the report is classified as Fun, the abstract is classified
as Jolly, and the SF-298 is classified as Amusing. Of course, actual 
classified documents should be classified with actual classifications.

\subsection{Labeling the paragraphs and caption}

(U) Each paragraph of your document should be proceeded with an
appropriate classification level. For example, this paragraph is
explicitly labeled as being unclassified. It appears in the source of
this document as this:

\begin{quotation}
\begin{Verbatim}[fontsize=\small]
(U) Each paragraph of your document should be proceeded with an
appropriate classification level. For example, this paragraph is
explicitly labeled as being unclassified. It appears in the source 
of this document as this:
\end{Verbatim}
\end{quotation}

As you can see from the example above, paragraph classification
labeling must be done manually. Captions must also be manually labeled.

\subsection{Labeling Your References}

When citing references in a classified document, use the
|\citeafter{}| macro instead of the |\cite{}| macro.

When you use BiB\TeX to produce a classified document you should use
either the bibliographic style |nps-plain-classified| or
|nps-plain-classified-unsorted|. These styles have been modified to
support an additional |classification| tag. Below, the |mm2| reference
is classified |F| and is further within the MM compartment:

\begin{Verbatim}
    @misc{mm2,
      title="Ears, Ears and More Ears",
      publisher="Department of Departments",
      author="Micky Mouse",
      year=2013,
      classification="(F//MM)"
    }
\end{Verbatim}

\section{Additional Files Included in the Template}
There are several files included with the template that may be useful for your writing needs.

\begin{description}
\item[Makefile] Included with the template is the |Makefile| that Mac and Linux users will readily enjoy.  Typing |make| on the command prompt will perform all necessary commands to produce your document.

\item[build.py] An alternate build system for Windows users.

\item[authorindex.pl] This perl script is used to generate the
  authorindex.  You will need to use this script if you are generating
  your document with the authorindex option (see
  \S\ref{thesisprologue}).  An additional install of a perl
  interpreter is required for Microsoft\textregistered{} Windows
  (ActivePerl\textregistered{} is recommended).

\item[fixerrors.py] This python script will correct |.bib| file entries for URLs that contain long URLs and also corrects errors in the authorindex |.ain| files.

\item[xls\_extract.py] This python script extracts all Excel terms from an NPS budget spreadsheet and write \LaTeX{} variables.
  Although it is unlikely you will need to use the script exactly, it can be a reference of how to do something similar if needed for your document.

\item[xls\_covert\_to\_pdf.py] Converts the Excel workbook to PDF
  file. This program requires additional software to operate properly
  and can only be used on a Macintosh computer.
\end{description}

\section{Additional Software}
This section discusses additional software that you may find useful
when preparing your document.
\subsection{Citation Management Software}
Organizing your thesis citations is critical to a successful thesis.  Legacy techniques included using index cards.
In modern times, software is available to help you accomplish this task.  A complete list of the available options is
at \url{http://en.wikipedia.org/wiki/Comparison_of_reference_management_software} .  NPS has a site-license for Refworks.
Other highly recommended options are Zotero and Mendeley.  See \url{http://www.zotero.org/} and \url{http://www.mendeley.com/}
for additional details.
   
\subsection{Revision Control Systems and Subversion}
Revision control software such as subversion (|svn|), mercurial (|hg)|, |git|, and others are excellent modern choices.  
Consult their websites to determine which one best suits your needs.

You will note that \LaTeX{} creates many temporary files. These files should \emph{not} be
included in your subversion repository. Because they are generated on a
per-machine basis, you can get conflicts if different files are
created and then committed on different machines.

If you are using subversion to manage your thesis, you should instruct it to ignore these files.  This
can be done with the \texttt{make ignore} target in the Makefile.

\begin{Verbatim}
ignore:
        svn propget svn:ignore . > /tmp/ignore
        echo thesis.pdf >> /tmp/ignore
        echo '*.ain' >> /tmp/ignore
        echo '*.aux' >> /tmp/ignore
        echo '*.asy' >> /tmp/ignore
        echo '*.bbl' >> /tmp/ignore
        echo '*.blg' >> /tmp/ignore
        echo '*.lof' >> /tmp/ignore
        echo '*.log' >> /tmp/ignore
        echo '*.lot' >> /tmp/ignore
        echo '*.sow' >> /tmp/ignore
        echo '*.toc' >> /tmp/ignore
        echo '*.zip' >> /tmp/ignore
        sort /tmp/ignore|uniq|grep .|svn propset svn:ignore -F - .
        @echo ""
        @echo Will ignore:
        svn propget svn:ignore .
        @/bin/rm -f /tmp/ignore
\end{Verbatim}



\section{Going Further}
If you are interested, feel free to review the file
|npsthesis.cls|. A great deal of effort has gone into making this
file both readable and understandable. You will find additional
commands in this file and you may even have thoughts on changes to
make. Please let us know what you come up with!



\chapter{Helpful Writing Tips for Your Report or Thesis}
This chapter discusses elements of writing and style that are helpful
when writing a report or thesis at NPS. This chapter is based on a
publication that was distributed by the NPS Thesis Processor in 2009.

\section{English Grammar Tips}
\begin{enumerate}
\item Punctuation (periods and commas) go inside quotation marks. 
\item When using \ie \eg or \etc always put
  a comma before and after, \emph{e.g.}, like this. You can also use
  the |\ie|, |\eg| and |\etc| macros that the thesis template provides.
\item Master's degree has an apostrophe and Postgraduate is one word. 
\item If you use ``however,'' make sure there's a comma before and after,
  unless you start a sentence with it. However, it's best not to start a sentence
  with ``however.'' And while we are on the subject, you should try to avoid 
  starting a sentence with ``and'' or ``because.'' 
\item When typing a date, do not use ``st'' or ``th.'' Instead, just
  note the date: July 4, 1776, is Independence Day. Commas go 
  after Month/date, year. No comma between month/yr. 
\item Spell out numbers 1 through 9 as one to nine.  Larger numbers remain as digits.  
\item Capitalize C in Chapter, F in Figure and T in Table when
  referring to chapters, figures or tables in the text and use roman
  numerals vs numbers or spelling out, etc. for chapters. Even
  better, use the referencing commands described in \secvref{refcommands}.
\item Footnote numbers go outside the punctuation. 
\item Ibid cannot be the first footnote on the page.
\item When typing equations in text and when using ``where'' or ``if,''
  \etc and it's not a new paragraph the word starts at the margin.
\item When inserting symbols, use the proper symbols commands.  Avoid trying to
  include the character directly in the |.tex| file.
\item Avoid writing in the first person!
\item Avoid dangling participles.  Wrong: Substituting (12) into (14) gives...;
  Correct: Substituting (12) into (14), we get...
\item Contractions are not used in formal writing.  Cannot is one word.
\item Chapters, figures and tables do not show things.  Instead, things are shown
  or illustrated in figures and tables.  Things are discussed in chapters or sections.
\end{enumerate}

\section{Additional Writing Tips}
\begin{enumerate}
\item Displayed equations must be numbered, part of a sentence and properly punctuated.  This means
  your equation may have a period as the last character to indicate the sentence has ended. 
\item In-line equations in paragraphs must be simple and use ``/'' to indicate division.
\item All figure captions should be complete sentences with a period at the end of the caption.
\item Figures and tables should display the units associated with quantities being displayed.
\item Axes in figures should be clearly labeled with quantities and units.
\item Discuss all figures and tables in your thesis.
\item Acronyms need to be defined in the acronyms list, in the abstract, in the executive
  summary, and the first time they are used in the thesis.  They do not need to be
  defined for every chapter.
\item The introduction should provide the background that allows the reader to
  understand why he or she should be interested in the problem.  Provide a
  discussion of related work with references.  State clearly and explicitly the
  goal(s) or objective(s) of your work.  Discuss how your work differs from the
  previous works.
\item Abstracts briefly summarize the work and help the reader to ascertain the purpose
  of the thesis.  An abstract may include the problem at hand, the technique used to solve
  the problem, and indicate the conclusion of the results.
\item Some departments require an an additional section called the Executive Summary.
  It is more comprehensive than the abstract and generally 2-10 pages in length.  The
  Executive Summary must stand alone from the rest of the document.  Figures and tables
  are numbered independently from the thesis content and do not appear in the List of
  Figures or List of Tables.  Additionally, references in the Executive Summary are
  independent from the thesis and there is a separate list of references at the
  end of the Executive Summary.   
\item Conclusions summarize the results obtained in your research and emphasize
  your original contributions.  Recommended future work should include any new
  questions arising from your research.
\end{enumerate}

\section{\LaTeX{} Tips}
\begin{enumerate}
\item Do not use ``*'' or ``x'' to indicate multiplication.  $X=YZ$ is sufficient.
\item If you must use multiplication, please do so using |\times| in
  math mode. That is, type |$X=Y\times Z$| to produce $X=Y\times Z$.
\item Use the \LaTeX{} \verb|\begin{figure}| and \verb|\begin{table}| environment to
  create floating figures and tables. Use the |\caption| command
  to create your captions. Label your captions with the
  |\label{marker}| command inside the caption itself. Captions are shown
  in the paper as text.  Labels are internal \LaTeX{} identifiers that can
  be referenced with the |\ref{marker}| reference command.
\item Do not split text around a figure or table.  Write complete paragraphs,
  since \LaTeX{} will place figures in your document to efficiently use the paper.  
\item When there is more than one reference, put them both into the 
  \verb|\cite| command: \\ \verb|\cite{john1,john2,john3}|.  The additional \texttt{cite} package
  can improve the multiple citations result from [6,7,8] to [6-8].
\item Make sure there's at least one and a half lines of text at the
  top of the page---if \LaTeX{} gives you a hard time, you may need to
  add or remove text so that everything works out properly.
\item Don't use math mode as a general italics---use |\emph{}|. 
\item Do not make tables too wide in columns or they can be drawn off the right-side of the paper.
\item Use automatic numbering and lettering by using the appropriate environments, such as
  \texttt{enumerate}, \texttt{itemize} or \texttt{list}.
\item There can only be one |label| entry for a section, figure, \etc.  Trying to have more than one will
  cause a problem in the automatic numbering.  If you need to troubleshoot numbering, look in the generated |.toc| files.
\end{enumerate}

\section{NPS Thesis Tips}

When writing your thesis, the sections should be in this order:
\begin{enumerate}
  \item Table of Contents
  \item List of Figures*
  \item List of Tables*
  \item List of Acronyms*
  \item Executive Summary*
  \item Acknowledgements*
  \item Chapters
  \item Appendices
  \item Author Index*
  \item Index*
  \item References
  \item Initial Distribution List
\end{enumerate}

The starred items do not appear in Table of contents.  Not all theses
will have an index, but that can be generated automatically with
\LaTeX{}.  If your thesis only has one appendix, then it is not
lettered, but just referred to as the appendix.  If you have multiple
appendices, then they are lettered.  If you do not have any tables,
then the list of tables is not included.  This also applies to list of
figures and acronyms.  The Executive Summary is required by the
Electrical Engineering, Systems Engineering, and Operations Research
departments.

\section{Initial Distribution List Recipients}
Each thesis contains a list at the end of the original recipients of the thesis.  You may 
add additional names to this list for professional or personal reasons.  When you submit your
thesis, you will provide an email address for the individuals on the list, and they will 
receive an email when the thesis has been posted.

\begin{enumerate}
  \item Defense Technical Information Center\\Ft. Belvoir, Virginia
  \item Dudly Knox Library\\Naval Postgraduate School\\Monterey, California
  \item Your department chair\\Naval Postgraduate School\\Monterey, California
  \item Your advisor\\Naval Postgraduate School\\Monterey, California
  \item Your 2nd reader/coadvisor\\Naval Postgraduate School\\Monterey, California
  \item Each author\\Naval Postgraduate School\\Monterey, California\\
    
  \textit{Marine officer students are required to show:}

  \item Marine Corps Representative\\Naval Postgraduate School\\Monterey, California
  \item Directory, Training and Education, MCCDC, Code C46\\Quantico, Virginia
  \item Marine Corps Tactical System Support Activity (Attn: Operations
    Officer)\\Camp Pendleton, California\\

  \textit{Officer students in the Operations Research Program are also required
    to show:}

  \item Director, Studies and Analysis Division, MCCDC, Code C45\\
    Quantico, Virginia\\

  \textit{Officer students in the Space Ops/Space Engineering Program or in the
    Information Warfare/Information Systems and Operations are also
    required to show:}

  \item Head, Information Operations and Space Integration Branch,\\
    PLI/PP\&O/HQMC, Washington, DC
\end{enumerate}

\section{Documentation Submitted with Thesis}
The Thesis Processors will require some additional forms to accept
your thesis and give you a green card.  These forms are available on
their website at \url{http://www.nps.edu/research/research1.html}.

\begin{itemize}
  \item Thesis release form
  \item Thesis color page request form
  \item Special Abstract
  \item Completed Signature Page
\end{itemize}

%\chapter{Introduction to Graphing with \LaTeX}
In this (brief) chapter, we show several approaches for constructing
bar graphs that present scientific information. This chapter is
designed to provide you with a starting point for looking at graphing
packages. We expect that you will use this chapter to learn about the
various options available, and then review the documentation
associated with that option once you have made a decision of how you
wish to proceed.

There are many, many options for presenting numeric information in
graphical form using \LaTeX. Fundamentally, though, all of the options
fall in one of two categories:

\begin{enumerate}
\item You can create the graphics entirely within \LaTeX.
\item You can create the graphics with a second package and include
  the graphics file in \LaTeX{} with an |\includegraphics| command.
\end{enumerate}

In this section, we will assume that you wish to make two graphs.

\begin{description}
\item[Graph \#1---Accumulated Spending] This graph will show the
  amount of money spent on a project from January through May, the
  total amount spent, and the total amount allocated for calendar
  year. The data for this graph is shown in \tabref{spending}.
\item[Graph \#2---Scatterplot of (x,y)] This graph will show ten
  values of (x,y) and a line that fits them.
\end{description}


% APPENDICES
% You have two recommended options for your appendix:
% a) A single appendix (with a single TOC entry)
% b) Multiple appendices. Look under the examples directory for a demo of
%   multiple appendices.
%

% REFERENCES
% List all your BibTeX reference source files (ending in *.bib extension)
%
\NPSbibliography{npsreport}


%
% This is the official end of the thesis.
%
\NPSend

% DISTRIBUTION LIST
% The list is automatically properly numbered
% and already populated with the mandatory recipients.
%
\NPSdistribution{Initial Distribution List}
\begin{distributionlist}
\item Defense Technical Information Center\\Ft. Belvoir, Virginia
\item Dudley Knox Library\\Naval Postgraduate School\\Monterey, California
\end{distributionlist}


\end{document}
