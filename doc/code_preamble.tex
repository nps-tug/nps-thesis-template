
\begin{lstlisting}
\documentclass[twoside,thesis]{npsreport}   (*@\label{code:documentclass}\comment{
			See \S\ref{sec:documentclass} for details.
}@*)

\title{[Title]} (*@\comment{Your thesis or report title.}@*)
\author{[Author Name]} (*@\comment{
			Your name. See examples for multiple authors.
}@*)
\rank{[Rank, Service]} (*@\comment{
			Your rank. If you are a civilian,\
			use {\upshape\texttt{\textbackslash rank\{Civilian, Organization Name\}}}.
}@*)
\degree{Master of Science in [Degree]}  (*@\comment{
			Your NPS degree, written out. See the examples for dual degree macros.
}@*)
\degreeabbreviation{MS}    (*@\comment{
			This should be either MS, MBA or MA.
}@*)
\prevdegrees{[B.S., My Old School, Year]} (*@\comment{
			Degree from your previous school. If you have more than one degree,\
			you can use the command \upshape{\texttt{\textbackslash prevdegrees\{B.S.,\
			Harvard, 1901\textbackslash\textbackslash M.S., Yale, 1904\}}}.
}@*)

\department{Department of [Department]} (*@\comment{
			The name of your academic department.
}@*)
\thesisadvisor{[Primary Advisor]} 
\secondreader{[Second Reader]}
\departmentchair{[Department Chair]} (*@\comment{
			The name of your department chair. Per NPS style, do not \
			use Dr., Prof., or similar titles.
}@*)
\degreedate{[Month Year]}   (*@\comment{
			The date you are graduating.
}@*)
\distribution{Approved for public release; distribution is unlimited}  (*@\comment{
			Review your thesis release form for approved distribution statements.
}@*)

\abstract{
  [INSERT ABSTRACT HERE] (*@\comment{
  			Your abstract goes here, or you can include\
  			data from another file, like\
  			\upshape{\texttt{\textbackslash input\{abstract.tex\}}}
}@*)
}

\securitybanner{} (*@\comment{
			Leave blank, unless FOUO or classified\
			(see \S\ref{sec:classified}).
}@*)
\ReportType{Master's Thesis} (*@\comment{
			Master's Thesis, Technical Report, or Dissertation
}@*)
\ReportDate{MM-DD-YYYY} (*@\comment{
			SF298: The final date of your report.
}@*)
\DatesCovered{2102-06-01 to 2104-10-31} (*@\comment{
			SF298: The range of dates relevant to your report.
}@*)
\SponsoringAgency{Department of the Navy}   (*@\comment{
			SF298: Your sponsoring organization, \eg,\
			Department of the Air Force, Department of the Navy.
}@*)
\ReportClassification{Unclassified}  (*@\comment{
			SF298: classification of report.
}@*)
\AbstractClassification{Unclassified} (*@\comment{
			SF298: classification of the abstract.
}@*)
\PageClassification{Unclassified} (*@\comment{
			SF298: classification of the SF298 form itself.
}@*)
\POReportNumber{} (*@\comment{
			SF298: for Technical Reports, the technical report number.
}@*)
\RPTpreparedFor{}
\ContractNumber{}
\GrantNumber{}
\ProgramElementNumber{}
\TaskNumber{}
\WorkUnitNumber{}
\Acronyms{}
\SMReportNumber{}
\SubjectTerms{}
\ResponsiblePerson{}
\RPTelephone{}

\SignatureOne{}
\SignatureTwo{} (*@\comment{
			Optional: Each author's signature line can include an image\
			of a signature; specifying the width as 2 inches is recommended.\
}@*)

\SupplementaryNotes{The views expressed in this document are those of
  the author and do not reflect the official policy or position of the
  Department of Defense or the U.S. Government.
  IRB Protocol Number: N/A    (*@\comment{
			If your thesis has an Institutional Review Board (IRB) protocol\
			number, replace N/A with the the appropriate IRB number given\
			to you or your advisor when your experiment was approved;\
			this is \emph{required} for theses that use human subjects to\
			collect data. Ask your advisor for more information if this\
			applies.
}@*)
}
\makeatletter (*@\label{code:note1}@*)
\@removefromreset{footnote}{chapter} 
\makeatother  (*@\label{code:note2}@*) (*@\comment{
			Lines~\ref{code:note1}--\ref{code:note2} prevent footnotes from\
			being reset at each chapter. Comment them out to have footnotes\
			reset with each chapter.
}@*)

\NPShyperref (*@\comment{
			This includes the {\upshape \texttt{hyperref}} package,\
			placing links and metadata in your final PDF.\
			If this causes conflicts, you might want to remove this command.\
			Note: because of the way metadata is embedded in the PDF,\
			you may have problems if macros or math notation are used\
			to define your Title, Author or SubjectTerms macros.
}@*)
\end{lstlisting}


\subsection{Document Class Options}\label{sec:documentclass}
The \verb|documentclass| command (see \S\ref{sec:prologue}, Line~\ref{code:documentclass}) 
specifies that the document uses the |npsreport.cls|
  file and all settings contained therein.  There are  several optional
  parameters, each separated by comma: 
  \begin{description}
    \item[article, thesis, or dissertation] Choose the appropriate one 
    for the case. For disserations, see \S\ref{sec:dissertation}.
    \item[12pt, 11pt, or 10pt] Font size selection.  With no option given, 12pt is the default.
    \item[times, arial, or courier]  Font selection.  With no option given, times is the default.
    \item[twoauthors, threeauthors, or fourauthors] Use these options if you have 
    several authors.  Single authors need no option. 
    \item[twoadvisors] If you have two advisors rather than a second reader. 
    \item[fivemembers,advisoralone] Dissertations by default have six committee members. 
    If you have only five members, use the |fivemembers| option; if you want the fifth member
    to be centered on the final line alone (which looks rather nice and symmetric), use the |advisorsalone| option.
    \item[twoside] Prints on both sides of the same sheet of paper; recommended.
    \item[classified] If you are using an approved computer system to 
    write your thesis on sensitive research.
    \item[index] If you are including a keyword index page of your 
    thesis important terms.
    \item[acronym] For a more sophisticated handling of acronyms.  See 
    |acronyms.tex| For additional information.
    \item[traditional] Prints the thesis in the style of the NPS
      Microsoft Word thesis template. Although you are free to use
      this style, the newer style is approved and looks quite nice 
    when no option is given.
    \item[singlespace] If you prefer single-spaced paragraphs, though it 
    may be a little harder to read. This is not approved for an NPS
    Masters Thesis, but is approved for NPS technical reports.
  \end{description}

\subsection{Multiple Authors and Multiple Advisors}
The template supports a variety of options:
\begin{itemize}
\item Many authors (2--4 authors)
\item One advisor and one second reader
\item Two co-advisors
\item One advisor and one co-advisor and one second reader
\item \etc
\end{itemize}
Each of these require some extra macros to be defined.
See examples for populating the appropriate macros.
